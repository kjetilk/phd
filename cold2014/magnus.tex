\documentclass{llncs}
%\documentclass{article}
\usepackage[utf8]{inputenc}
\usepackage{cite}
\usepackage{verbatim}
\usepackage{graphicx}
\usepackage{wrapfig}

\title{Performance of two different ontological and data access methods}
\author{Magn\'{u}s D\ae hlen\inst{1} \and Kjetil Kjernsmo\inst{1}
\institute{Department of Informatics,
Postboks 1080 Blindern,
N-0316 Oslo, Norway \email{\magnudae,kjekje\}@ifi.uio.no} 


\subtitle{---Draft submitted to COLD 2014---}


\begin{document}
\maketitle

\begin{abstract}
  We evaluate two different approaches that have been used in the
  literature and in practice by different groups publishing metadata
  about cars when consumed by a prototype application. These
  approaches are characterized by a generic vs. domain specific
  ontology; by a constrained API vs. openly queryable data % TODO:
                                % Magnus, what else
  We have implemented a prototype application where a potential user
  may query selected properties of cars, and we investigate how the
  different choices made by the original designers influence the
  performance of the application. For the evaluation, we employ
  the statistical disipline of Design of Experiments. 

\end{abstract}


\end{document}
