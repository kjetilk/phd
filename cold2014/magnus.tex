\documentclass{llncs}
%\documentclass{article}
\usepackage[utf8]{inputenc}
\usepackage{cite}
\usepackage{verbatim}
\usepackage{graphicx}
\usepackage{wrapfig}

\title{Performance of two different ontological and data access methods}
\author{Magn\'{u}s D\ae hlen\inst{1} \and Kjetil Kjernsmo\inst{1}
\institute{Department of Informatics,
Postboks 1080 Blindern,
N-0316 Oslo, Norway \email{\magnudae,kjekje\}@ifi.uio.no} 


\subtitle{---Draft submitted to COLD 2014---}


\begin{document}
\maketitle

\begin{abstract}
  We evaluate two different approaches that have been used in the
  literature and in practice by different groups publishing metadata
  about cars when consumed by a prototype application. These
  approaches are characterized by a generic vs. domain specific
  ontology; by a constrained API vs. openly queryable data % TODO:
                                % Magnus, what else
  We have implemented a prototype application where a potential user
  may query selected properties of cars, and we investigate how the
  different choices made by the original designers influence the
  performance of the application. For the evaluation, we employ
  the statistical disipline of Design of Experiments. 

\end{abstract}

\section{Introduction}

The retrieval of information from different sources and then combine
them to allow a user to find a certain combination of properties that
suit their purpose is an archetypical Semantic Web use case. We have
chosen to focus on a product selection use case, specifically on cars,
since some car makers have embraced the Semantic Web vision and chosen
to share detailed data on their products.

The findings, we assert, have a broader validity, given the generic
nature of the problem they are trying to solve, and so our
recommendations should have broad relevance to similar use cases.

We present two different approaches, one promoted by Renault, see
\cite{SemWebAppRes} and \cite{ren1}, and another made by Martin~Hepp
in collaboration with Volkswagen resulting in amongst other things,
the Car Options Ontology~\cite{COO}. We shall compare them to find
their strengths and weaknesses. Each approach contains an ontology and
has its own way of representing data.  To make this comparison we have
used data about the same domain, data about car models and their
component constraints.  The first approach is a generic ontology. By
generic it means that it can be used to represent any product model
with component constraints. The second approach is a domain specific
ontology which as the name implies, is only applicable with one
particular domain. In this thesis that domain is about car models.

We will also show how to create a viable web application which
utilizes such complex data, mainly for the purpose of conducting
performance tests to determine the weaknesses of each approach. This
will be done with complex data found on the web today from different
car manufacturers.  With performance we mean the response time between
a HTTP~\cite{http}} post operation against the application and when
the application presents the user with an answer. The application will
contain the possibility to do HTTP posts against both approaches.

There will be several options on how to query the data because of all
the different specifications.  That is why we have chosen to use the
testing approach \textit{Design of Experiments} (DoE). This approach
will be further explained alongside the results in
Section~\ref{Results}. The evaluation will be based on four
experiments testing several aspects of the approaches. They will help
us determine what kind of factors are significant to the performance.

\section{Related work}
Complex products and specifying configurations has been a research
topic for over a decade.  The possibility to personalize more and more
products is why several research articles have proposed different
approaches on how to handle these configurations. Most of the research
are around finding the ultimate solution with a specification
system. In 1999, M. Aldanondo et. al proposed how to structure a
system to handle configurations and their constraints. This included
proposed definitions for products, configurations and
configurators. The paper focused on making a generic solution to fit
several manufacturers.~\cite{OldConf} In a newer article,
H. Afsarmanesh and M. Shafahi (2013) proposed a complex product
specification system.~\cite{NewConf} This include object modelling and
a user interface. They have focused on supporting stakeholders in the
specification process.

Unfortunately these articles do not present any research done with
semantic technologies. The use of semantic technologies on complex
products is a young field of research. The only thing done here is
what Renault and Volkswagen have presented. Both of these car
manufacturers have presented the public with two different
solutions. They have also shown how to present the data on the web and
the complexity around their solutions.



\end{document}


