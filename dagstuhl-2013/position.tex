\documentclass{article}
\usepackage[utf8]{inputenc}
\usepackage{hyperref}
\usepackage{cite}
\usepackage{verbatim}
\usepackage{graphicx}

\title{Merits of hypermedia systems}
\author{Kjetil Kjernsmo}

\begin{document}

\maketitle

\section{Research interests}

My primary research interest is the optimisation of SPARQL queries in
a federated regime, as we have noted that this is not practical
because the federation engine has insufficient information to
optimise, or the information is so large that it defeats the purpose
of optimisations to begin with. I plan to help remedy this problem by
computing very compact digests and expose them in the service
description. I have not yet published any articles on this topic, but
the research is in the immediate extension of
SPLENDID\cite{splendid}. My secondary research interest is replacing
the practice of benchmarking by using statistical design of
experiment. 

However, coming from an industry background in software development,
experience suggests that the above research interests does not
adequately address many immediate needs when developing information
systems to process the rapidly increasing amount of available data. I
believe that large SPARQL-driven systems would be the ``right tool for
the job'' in only a limited, and currently unclear, set of
cases. Further exposure to new ideas in the developer community lead
me to develop a third interest, namely hypermedia RDF. 

Problems with SPARQL interfaces are many: They require extensive
training of developers; it is not immediately clear what data are
available and what may done with the data; it is easy to formulate
queries that will cause the endpoint to become overloaded and hard to
protect against them without also rejecting legitimate queries; it
places heavier systems in the execution path of an application, etc.

\section{Hypermedia RDF}

In \cite{kjernsmo_lapis_2012}, I examined some practical implications of
the HATEOAS constraint of the REST architectural style, see
\cite{Fielding_2000_Architectural-Styles}~Chapter~5., and in that
light argued why hypermedia RDF is a practical necessity.

Mike Amundsen defines hypermedia types\cite{hypermediatypes} as 
\begin{quote}
Hypermedia Types are MIME media types that contain native
hyper-linking semantics that induce application flow. For example,
HTML is a hypermedia type; XML is not.
\end{quote}

We continued to derive a powerful hypermedia type based on RDF within
a classification suggested by Mike Amundsen. Since this publication, I
have also noted that the embedded links factor can be achieved by
using \texttt{data} URIs, thus satisfying all but one of the factors
proposed by Mike Amundsen.

Further, we noted that other interesting factors is the
self-description, which is a important characteristic of the RDF
model, and other minor concerns.

To bring forward a concrete example of how to make a serialised RDF
graph into a hypermedia type, we suggest adding some triples to every
resource (where prefixes are omitted for brevity):

\begin{verbatim}
<> hm:canBe hm:mergedInto, hm:replaced, hm:deleted ;
   hm:inCollection <../> ;
   void:inDataset [ void:sparqlEndpoint </sparql> . ] .
\end{verbatim}

\section{Possible uses}

Over time, I believe that interfaces that require developers to read
external documentation will loose to interfaces where ``View Source''
is sufficient to learn everything that is needed. This is the essence
of hypermedia systems, in the RDF case, everything needed to program
is available in the RDF. It tells application what it may do next.

In the above, we have included mostly create, add and delete
primitives, but these could be refined for application scenarios if
needed.

For example a pizza baker publishes Linked Data about pizzas they
sell, including data sufficient to create a sophisticated search and
sales application. The Linked Data will then include triples that
state explicitly how the order should be placed, i.e. what resources
need updating. Moreover, the may not only want to sell pizzas, but
also drinks. Thus, the Linked Data presented to the application should
not only be a Symmetric Concise Bounded Description, as common today,
but careful designed to provide exactly the data needed to optimise
sales.

\bibliographystyle{plain}
%\bibliographystyle{jbact}
%\bibliographystyle{splncs03}
\bibliography{federation,webarch,vocabs,specs,hypermedia,egne,selectivity}

\end{document}
