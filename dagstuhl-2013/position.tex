%\documentclass{article}
\usepackage[utf8]{inputenc}
\usepackage{cite}
\usepackage{verbatim}
\usepackage{graphicx}

\title{Merits of hypermedia systems}
\author{Kjetil Kjernsmo}

\begin{document}

\maketitle

\section{Research interests}

My primary research interest is the optimization of SPARQL queries in
a federated regime, as we have noted that this is not practical
because the federation engine has insufficient information to
optimize, or the information is so large that it defeats the purpose
of optimizations to begin with. I plan to help remedy this problem by
computing very compact digests and expose them in the service
description. I have not yet published any articles on this topic, but
the research is in the immediate extension of
SPLENDID\cite{splendid}. My secondary research interest is replacing
the practice of benchmarking by using statistical design of
experiment. 

However, coming from an industry background in software development,
experience suggests that the above research interests does not
adequately address many immediate needs when developing information
systems to process the rapidly increasing amount of available data. I
believe that large SPARQL-driven systems would be the ``right tool for
the job'' in only a limited, and currently unclear, set of
cases. Further exposure to new ideas in the developer community lead
me to develop a third interest, namely hypermedia RDF. 
