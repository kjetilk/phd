\documentclass{llncs}
%\documentclass{article}
\usepackage{cite}

\title{Introducing Statistical Design of Experiments to SPARQL
  Endpoint Evaluation}
\titlerunning{Introducing Design of Experiments to SPARQL Evaluation}
\author{Kjetil Kjernsmo\inst{1}}
\institute{Department of Informatics,
Postboks 1080 Blindern,
0316 Oslo, Norway
\email{kjekje@ifi.uio.no}}


\begin{document}

\maketitle

\begin{abstract}
This paper argues that the common practice of benchmarking is
inadequate as a scientific evaluation methodology. It further attempts
to introduce the empirical tradition of the physical sciences by using
techniques from Statistical Design of Experiments applied to the
example of SPARQL endpoint performance evaluation. It does so by
studying full as well as fractional factorial experiments designed to
evaluate an assertion that some change introduced in a system has
improved performance. This paper does not present a finished
experimental design, rather its main focus is didactical, to shift the
focus of the community away from benchmarking towards higher
scientific rigor.
\end{abstract}

\section{Introduction}

\section{Related work}

\section{Conclusion}

\section*{Acknowledgements}




\end{document}
