%\documentclass[english,aspectratio=169]]{ifislide}
%\documentclass[english,print,aspectratio=169]{ifislide}
\documentclass[english,handout,aspectratio=169]{ifislide}

\usepackage[utf8]{inputenc}
% Packages used specific for this lecture:

%\usepackage{bussproofs}
%\usepackage{textpos}

%\EnableBpAbbreviations
%% \usepackage{tikz}

%% \usetikzlibrary{arrows,shapes,mindmap}
%% \usetikzlibrary{decorations.pathmorphing}
%% \tikzstyle{resource}=[ellipse,draw=black!40,fill=blue!20,thick, font=\ttfamily]
%% \tikzstyle{blank}=[ellipse,draw=black!40,fill=white!40,thick, font=\ttfamily]
%% \tikzstyle{literal}=[rectangle,text centered,draw=black!40,fill=black!20,thick]
%% \tikzstyle{predicate}=[sloped,above,text centered,midway,font={\scriptsize\ttfamily}]
%% \tikzstyle{predicateb}=[left,midway,font={\scriptsize\ttfamily}]
%% \tikzstyle{class}=[rectangle,text centered,draw=black!40,fill=white,thick]
%% \usetikzlibrary{arrows,automata}


\newcommand{\typed}{\char`\^\char`\^}
\newcommand{\blank}{\char`\_}

\newcommand{\tuple}[1]{\left<#1\right>}
\newcommand{\inter}{\mathcal{I}}
\newcommand{\domain}{\mathop{\mathsf{dom}}}
\newcommand{\range}{\mathop{\mathsf{rg}}}

% set lecture number, date, title:
%\LectureNumber{13}
\LectureDate{23.10.2013}
\LectureTitle{Introducing Statistical Design of Experiments to SPARQL Endpoint Evaluation}
% set name of lecturer:
\author{Kjetil Kjernsmo}
\AtBeginSection[]
{
  \begin{frame}
    \frametitle{Outline}
    \tableofcontents[currentsection]
  \end{frame}
}

\begin{document}

\IfiTitleSlide     % inserts title frame

%\IfiTOCSlide       % inserts today's plan frame


\section{Introduction} 

\begin{frame}{Experimental tradition}

Teh Tycho

\end{frame}

\begin{frame}{What's a benchmark?}
\end{frame}
\note{If your queries are all fully known, by all means, run them}
\note{In most scientific relevant cases, no such assumptions can be
  made}

\begin{frame}{Problems with benchmarking}

\begin{itemize}
\item Large number of parameters to test for complex scenarios.
\item No structured approach to investigate flaws.
\item No meaningful summary of the overall performance.
\item Standardized benchmarks cannot test assertions outside of the
  standard.
\item Attempting to neutralize the effect of certain optimization
  techniques oversimplifies the test.
\end{itemize}
\end{frame}
\note{If you standardize hardware, you can't test claims about
  improved hardware. You'd need interactions too}
\note{Cache-buster is evil}

\begin{frame}{Design of Experiments}

  \begin{itemize}
  \item Pioneered by Fischer in the 1930-ties.
  \item Well-established in many fields of engineering, agriculture,
    medicine.
  \item Well-suited to manage very complex experiments.
  \item Requires parameterization of experiments.
  \end{itemize}
\end{frame}
\note{How many is familiar with DoE?}

\subsection{Key Concepts Design of Experiments}

\begin{frame}{Response variable}

% TODO Format well
   A \emph{response variable} is measured under various
    combinations of parameters.

For example:
\begin{itemize}
\item Throughput
\item Query execution time
\item Response time
\end{itemize}

\end{frame}

\begin{frame}{Factor}

Such parameters are called \emph{factor}s. 

For example:
\begin{itemize}
\item Hardware platform
\item Data heterogeneity
\item Number of triples in the store
\item Absence or presence of certain language features
\item Order of experiments
\end{itemize}

\end{frame}

\begin{frame}{Levels}

For each factor, a range of possible values are fixed. These values
are called \emph{levels}.

For example:
\begin{itemize}
\item For number of triples, the levels could be 1 or 2 MTriples.
\item For language features, the levels could be absence or presence
  of an \texttt{OPTIONAL} pattern. 
\end{itemize}

Levels may be continuous, discrete, two different instances of a
class, etc.

\end{frame}

\begin{frame}{Design Matrix}
% latex table generated in R 2.15.1 by xtable 1.5-6 package
% Fri Oct  4 22:47:01 2013
\begin{table}[ht]
\begin{center}
\begin{tabular}{r|rrrrrrrr}
  \hline
 & ``Implement'' & ``TripleC'' & ``Machine'' & ``BGPComp'' & ``Lang'' & ``Range'' & ``Union'' & ``Optional'' \\ 
  \hline
1 & 2 & 2 & 1 & 2 & 1 & 2 & 2 & 2 \\ 
  2 & 1 & 1 & 1 & 1 & 1 & 1 & 1 & 1 \\ 
  3 & 1 & 2 & 1 & 2 & 2 & 2 & 1 & 2 \\ 
  4 & 1 & 1 & 2 & 1 & 2 & 2 & 1 & 1 \\ 
  5 & 1 & 1 & 1 & 2 & 1 & 2 & 2 & 2 \\ 
  6 & 1 & 2 & 1 & 2 & 2 & 1 & 1 & 1 \\ 
  7 & 1 & 1 & 2 & 2 & 1 & 2 & 2 & 1 \\ 
  8 & 2 & 1 & 2 & 2 & 2 & 2 & 1 & 2 \\ 
 ...  \\
  256 & 1 & 1 & 2 & 2 & 1 & 1 & 1 & 1 \\ 
   \hline
\end{tabular}
\end{center}
\end{table}
\end{frame}


\begin{frame}{Full normal plot}
\begin{figure}[t]
  \centerline{%
  \includegraphics[width=.9\textwidth]{fullnormal.pdf}}
  \caption{A normal plot of the Full Factorial Experiment. The
    labelled points are considered significant by using the Lenth
    criterion at a level $\alpha=0.05$.}\label{fig:fullnormal}
\end{figure}

\end{frame}
\note{Don't worry about the axes}


\begin{frame}
\begin{table}[ht!]
\begin{center}
\caption{$p$-values for different parts of the experiment}\label{tab:pvaluesfull}
\begin{tabular}{cccccl}
  \hline
``TripleC'' & ``BGPComp'' & ``Lang'' & ``Union'' & ``Optional'' & $p$ \\ 
  \hline
  1 & 1 & 1 & 1 & 1 & 0.012 \\ 
  1 & 1 & 1 & 1 & 2 & $1.4 \cdot 10^{-09}$ \\ 
  1 & 1 & 1 & 2 & 1 & $3.1 \cdot 10^{-09}$ \\ 
  1 & 1 & 1 & 2 & 2 & $6.1 \cdot 10^{-11}$ \\ 
  1 & 1 & 2 & 1 & 1 & $3.3 \cdot 10^{-06}$ \\ 
  1 & 1 & 2 & 1 & 2 & $2.7 \cdot 10^{-09}$ \\ 
  1 & 1 & 2 & 2 & 1 & $2 \cdot 10^{-06}$ \\ 
  1 & 1 & 2 & 2 & 2 & $3.6 \cdot 10^{-10}$ \\ 
  1 & 2 & 1 & 1 & 1 & 0.014 \\ 
  1 & 2 & 1 & 1 & 2 & $1.2 \cdot 10^{-10}$ \\ 
  1 & 2 & 1 & 2 & 1 & $2.8 \cdot 10^{-14}$ \\ 
  1 & 2 & 1 & 2 & 2 & $4.1 \cdot 10^{-15}$ \\ 
  1 & 2 & 2 & 1 & 1 & $2.1 \cdot 10^{-05}$ \\ 
  1 & 2 & 2 & 1 & 2 & $2.7 \cdot 10^{-07}$ \\ 
  1 & 2 & 2 & 2 & 1 & 0.0072 \\ 
  1 & 2 & 2 & 2 & 2 & $1.6 \cdot 10^{-05}$ \\ 
  2 & 1 & 1 & 1 & 1 & 0.28 \\ 
  2 & 1 & 1 & 1 & 2 & $3 \cdot 10^{-07}$ \\ 
  2 & 1 & 1 & 2 & 1 & $3.3 \cdot 10^{-07}$ \\ 
  2 & 1 & 1 & 2 & 2 & $1.7 \cdot 10^{-08}$ \\ 
  2 & 1 & 2 & 1 & 1 & 0.0023 \\ 
  2 & 1 & 2 & 1 & 2 & $6.5 \cdot 10^{-07}$ \\ 
  2 & 1 & 2 & 2 & 1 & 0.00032 \\ 
  2 & 1 & 2 & 2 & 2 & $1.3 \cdot 10^{-06}$ \\ 
  2 & 2 & 1 & 1 & 1 & 0.013 \\ 
  2 & 2 & 1 & 1 & 2 & $1 \cdot 10^{-11}$ \\ 
  2 & 2 & 1 & 2 & 1 & $3.8 \cdot 10^{-11}$ \\ 
  2 & 2 & 1 & 2 & 2 & $4.1 \cdot 10^{-15}$ \\ 
  2 & 2 & 2 & 1 & 1 &   1 \\ 
  2 & 2 & 2 & 1 & 2 & $2.3 \cdot 10^{-05}$ \\ 
  2 & 2 & 2 & 2 & 1 &   1 \\ 
  2 & 2 & 2 & 2 & 2 & 0.99 \\ 
   \hline
\end{tabular}
\end{center}
\end{table}

\end{frame}


\end{document}
