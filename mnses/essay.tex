
I mars 2006 fikk jeg en mail der jeg ble bedt om å komme til London
for et jobbintervju påfølgende helg. I denne bransjen foregår selv
jobbintervjuer under taushetserklæring, men essensen var denne:
``Gründerne har nettopp solgt sitt forrige prosjekt for 2 milliarder
euro, og er klare for å revolusjonere TV-mediet. De skal gjøre det på
medieindustriens premisser, etter tidligere å ha vært i konflikt med
disse.'' Etter en prat ble jeg forespeilet at jeg kunne jobbe fra hvor
som helst, når som helst, og med den teknologien jeg selv syntes mest
interessant (og som jeg nå skriver mitt doktorgradsarbeid om), ``bare
si din pris''. Jeg avslo tilbudet.

I dette essayet skal jeg forklare hvorfor de premissene
medieindustrien legger er i vil ifrata befolkningen viktige friheter,
stanse teknologisk utvikling, gi et inntrykk i befolkningen at
vitenskapen er autoritær og ekskluderende, og dermed gjøre at
vitenskapelig virksomhet faller i vanry. Videre har dette vide
implikasjoner for ytringsfrihet og demokrati. 

Dog ligger det i sakens natur at jeg som teknolog og naturviter kun
har overfladisk forståelse av de legale rammeverk som diskuteres, og
derfor vil jeg også beskrive tvil og problemer med bildet jeg tegner.

\paragraph{Relevante menneskerettigheter og min forståelse}

Den universelle menneskerettighetskonvensjonen\cite{udhr} er et
overordnet rammeverk for menneskerettigheter som er videre behandlet i
en rekke internasjonale eller overnasjonale avtaler. Dette er en
omfattende literatur jeg har lite grunnlag for å kjenne i detalj, og
jeg tar derfor utgangspunkt i konvensjonen.

I artikkel 29, 2. punkt kan man lese  
\begin{quote}
In the exercise of his rights and freedoms, everyone shall be subject
only to such limitations as are determined by law solely for the
purpose of securing due recognition and respect for the rights and
freedoms of others and of meeting the just requirements of morality,
public order and the general welfare in a democratic society.
\end{quote}

Det er min tolkning at begrensninger i individets frihet i form av lov
kun kan gjøres gjeldende der det er et klart og godt begrunnet behov
for dette. Videre må personene det gjelder faktisk ha mulighet til å
vite hva loven sier og hvordan man skal oppføre seg for å ikke bryte
den.

Artikkel 27 har to deler:
\begin{quote}
\begin{enumerate}
\item Everyone has the right freely to participate in the cultural
  life of the community, to enjoy the arts and to share in scientific
  advancement and its benefits.

\item Everyone has the right to the protection of the moral and
  material interests resulting from any scientific, literary or
  artistic production of which he is the author.
\end{enumerate}
\end{quote}

Opphavsrettslovgningen, i sine forskjellige former, skal gi den
beskyttelsen del 2 krever, mens del 1 har såvidt jeg vet ingen
eksplisitt beskyttelse. Dog begrunnes ofte sikring av forfatternes
materielle rettigheter med at det er det som gjør at befolkningen har
tilgang til verk.

\paragraph{Bevisbasert avgrensning}

Ettersom begrensninger i individets frihet må være klart begrunnet, og
et forbud mot eksemplarframstilling er en slik begrensning, så må vi
spørre om en del av utviklingen i senere år viser områder der slike
begrensninger er unødvendig.

Det er her naturlig å peke på Wikipedias suksess. Anekdotisk bevis
viser at Wikipedia kan vurderes på linje med etablerte leksika, se
omtale~i~\cite{naturewiki}. Hvis man godtar at Wikipedia er bra nok,
så må man så spørre seg om dette kunne vært oppnådd uten de
begrensninger åndsverksloven gir. Da må vi notere oss at Wikipedia
også har lisensbetingelser, men disse regulerer kun forhold som er
relatert til dagens opphavsrettsregime, de ville ikke være nødvendige
i et regime der leksika ikke hadde vern, ei heller regulerer de
eksemplarframstilling. Dermed bør Wikipedia tjene som et eksempel på
at begrensninger i befolkningens frihet når det gjelder leksika
\emph{ikke} er nødvendig.

Vi ser det samme i mange andre felter, både innenfor informasjon, data
og programvare. Man kan idag fint leve uten programvare som i et
regime der programvare ikke var vernet likevel ville eksistert i sin
nåværende form. Dette er programvare som gjerne benevnes som ``fri
programvare'' . Jeg bruker selv nesten utelukkende fri programvare, og
utgir selv det jeg skriver som fri programvare, og man kan finne alt
fra meget enkle programmer til meget avanserte systemer for
resonnering. Blant web-servere så er det to slike systemer som
konkurrerer i popularitet med Microsofts begrensende
tilbud\cite{netcraft}. Det er min oppfatning at dette betyr at
programvare i hovedsak ikke bør kunne underlegges begrensninger.

Dog ser jeg at det kan finnes unntak fra denne regelen. Jeg ser det
som utvilsomt at det er enklere å tjene penger på programvare som er
underlagt begrensninger basert på egen erfaring. Menneskerettighetene
sier ikke at det skal være enkelt å tjene penger, men det kan tenkes
at det blir så vanskelig å få til en investering i f.eks. veldig
sektor-spesifik programvare at ingen vil ta seg råd til å utvikle
den. Dette vil dermed gjøre at man ikke får til teknologisk utvikling.

Vi må videre spørre, gitt suksessen til Wikipedia og til fri
programvare, hvilke andre områder er det det er illegitimt å begrense
befolkningens frihet? Hvilke andre områder vil man kunne ha samme eller
forbedret kunstnerisk og vitenskapelig framgang uten slike begrensninger?

Jeg foreslår ikke en radikal avskaffelse av opphavsretten,
hovedpoenget er at enhver begrensning må begrunnes, og at man må være
svært forsiktig med å legge gjennomgripende begrensninger. Ettersom
enhver begrensning er en inngripen i befolkningens frihet, er det
naturlig at det stilles beviskrav til den som foreslår en begrensning.  

\paragraph{Inngripende tekniske tiltak}

Hittil har jeg diskutert kun klassisk opphavsrett, og medieindustriens
rolle er her begrenset. For å forsto hvorfor det ble så viktig for meg
å avslå et åpenbart godt tilbud, så må vi se til straffelovens
§262. Denne paragrafen i seg selv er totalt uforståelig for meg sett i
isolasjon, men forarbeidene til Åndsverksloven som kom noen år senere
gir et bedre innblikk i hva jeg antar det er snakk om, se spesielt
Ot.prp. nr. 46 2004-2005, seksjon 3.5.1 \footnote{se
  \url{http://www.regjeringen.no/nb/dep/kud/dok/regpubl/otprp/20042005/otprp-nr-46-2004-2005-/3/5.html?id=396451}}. 
Disse forarbeidene ser ut til å være noe mer nyansert enn
Straffeloven, men man ender opp med å gi lovbeskyttelse til
``effektive tekniske beskyttelsessystemer''. For å forstå hva dette er
henvises det til et EU-direktiv fra 1997, men i store trekk er det
systemer som kan brukes til å begrense tilgangen til et verk også
etter at man har tilegnet seg det, lovlig eller ulovlig. Jeg har ikke
lest EU-direktivet, fordi allerede i forarbeidene bærer det så galt av
sted at det for meg er åpenbart at lovreguleringen er av en slik natur
at det er umulig for befolkningen å vite om man bryter loven eller
ikke, uansett om hvor godt man kjenner lovgivningen. Denne tvilen om
hva som er lov og ikke er absolutt destruktiv. 

For å forstå mer av hva disse ``beskyttelsessystemene'' faktisk er, la
oss se på hva medieindustrien har kommet med.  I juli 2009 fjernet
Amazon e-bøker fra kundenes lesebrett selv om kundene hadde legitimt
kjøpt disse bøkene, se f.eks.~\cite{nytamazon}. Ironien er at blant
bøkene de fjernet var George Orwells ``Big Brother''. I denne boka har
myndighetene lagd et ``memory hole'' som brukes til å skrive om
historien. Essensen til tekniske beskyttelsessystemer er at man
tillater programvare på sin egen datamaskin, lesebrett eller
mobiltelefon, som man selv ikke har noen autoritet over. Denne
autoriteten beholdes av noen andre, og man har selv ingen kontroll på
hva de kan eller vil bruke denne autoriteten til. I tilfellet over er
det brukt til nettopp å lage Orwells ``memory hole'', og per idag har
altså lovgiver ment at slik bør det være. Vi har idag ingen sikkerhet
mot misbruk overhodet. 


Det mest skremmende som har kommet er en serie med dokumenter skrevet
av Motion Picture Association of America og adressert til det
amerikanske Senatet, med tittel ``Content Protection Status
Report''\footnote{Documentene var inntil nylig tilgjengelig fra
  Senatets websider, men nå finner jeg kun en omtale på Wikipedia:
  \url{http://en.wikipedia.org/wiki/Content_Protection_Status_Report}.
I disse dokumentene ble det foreslått tiltak som at all elektronikk
som ble lagd med analog-til-digitale konvertere måtte utstyres med noe
som gjorde at de kunne slås av hvis det ble detektert at de ble brukt
til å ta opp vernet verk. Bakgrunnen for forslaget er at man ønsker å
gjøre det umulig for uautoriserte å lage en mikrofon som kan brukes
til å gjøre digitale opptak. Grunnen til at dette er så skremmende er at
det ville føre til at man lagde en infrastruktur der Hollywood ville
fått absolutt autoritet over all produksjon. Dette er heldigvis ikke
gjennomførbart, men det at det ble presentert og tatt seriøst viser
hvor langt man er villig til å gå. 

Jeg mener at befolkningen ikke skal trenge å godta slike inngripen i
sin frihet. At man skal overgi autoritet over systemer som er så
personlige som en mobiltelefon til en tredjepart, som i essens kan
gjøre hva en vil med den, er for meg et sterkt inngripen i en persons
integritet og frihet. I essens handler dette om myndighetene skal ha
tillit til den enkelte borger. For å illustrere dette klarere, kan man
ta ``TV-bokser'' som idag formidles av blant annet GET. De siste
boksene fungerer både som modem og TV-dekoder. Dette vil føre til at
boksen som helhet er beskyttet av Straffelovens §262, hvilket igjen
betyr at man som borger ikke har noen praktisk mulighet til å
undersøke hva boksen i sin egen stue faktisk gjør. Det betyr at
produsenten av disse boksene har mulighet til å overvåke til minste
detalj hva brukerne gjør, både på nett og hva de ser på på TV. Det vil
kunne gjøres på en slik måte at det i praksis er umulig å oppdage uten
å muligens bryte Straffelovens §262, og jeg vil anta at overvåkningen
kan gjøres slik at veldig få vil trenge å vite om den, inkludert ingen
i GET eller andre selgere. Hvis dette scenariet ikke er skremmende
nok, så kan det nevnes at ifølge Wikipedia\cite{ndswiki}, så er firmaet som lager
disse boksene et selskap i Rupert Murdochs imperium og ledet av en
tidligere etterretningsoffiser.

Det blir derfor legitimt å spørre hvor norske politikere skal ha
større tillit til et medieimperium som beviselig har ulovlig overvåket
flere titalls mennesker\cite{newscorpphone}. Det er per idag ingen
indikasjoner på at dette faktisk skjer, men det viktige
tillitsspørsmålet bør være tverrpolitisk interessant. 

Det er videre viktig å merke seg at i alle tilfeller der befolkningen
som forbrukere har et reelt valg så velger de å ikke kjøpe produkter
med slike begrensninger. Den siste CDen med ``kopisperre'' kom ut i
2006\cite{aftkopi}, mens iTunes åpnet for å selge musikk uten sperrer året etter,
mens alle sperrer ble fjernet i 2009\cite{ituneskopi}, da det ikke gikk lenger å
konkurrere med slike sperrer. På DVD og TV har
ikke befolkningen noen reell valgmulighetet, og de består dermed.

Man må derfor stille spørsmålet: Når befolkningen som forbrukere med
stort flertall har forkastet et tiltak, skal det da være mulig i et
demokrati at politikerne opprettholder tiltaket når tiltaket begrenser
befolkningens frihet? Etter mitt skjønn burde det være politisk
umulig; fra et sosialdemokratisk perspektiv, der politikken anses å
være en mulighet for å korrigere markedskreftenes negative innflytelse
bør det være umulig å vedta og opprettholde lover som har større
negativ innflytelse på befolkningen enn markedet, og fra et
konservativt perspektiv, som vektlegger individets frihet, bør man
stille seg negativt til begrensninger som har vist seg unødvendige.

La oss så returnere til at jeg sa nei til jobbtilbudet: Den
umiddelbare grunnen bør dermed være klar: Det var klart at uansett
hvor revolusjonerende systemet var og hvor teknologisk interessant det
var, så ville det aldri bli noe av, ettersom folk rett og slett ikke
vil ha systemer som spiller etter underholdningsindustriens regler
hvis de har reelle valg. Jeg var ikke interessert i å drive med ting
som aldri vil bli noe av. Den ideelle grunnen bør også være klar, jeg
ville ikke være med på å ta fra befolkningen rettigheter. Prosjektet
kom imidlertid aldri så langt at det ble eksponert for store
brukergrupper, det kollapset på at mediene uansett ikke ville være med
på en revolusjon. Jeg har flere venner og bekjente som ble med, og de
satt i månedsvis på høye lønninger uten å gjøre noe annet enn å vente
på avklaringer.

\paragraph{Forskningen når ikke fram}

En del av folkene som hadde jobbet i dette prosjektet returnerte til
akademia og vant et stort EU-prosjekt kalt
NoTube\footnote{se \url{http://notube.tv/}}. Et av temaene for
prosjektet var integrasjon av sosial web med TV-mediet. Det kan være
så enkelt som at noen kommenterer på Twitter samtidig med at man ser
på en sending, og dette er i dag vanlig. Men prosjektet gir også
åpninger for langt kraftigere bruk: Gjennom åpne mediesenter-systemer
kan man legge til systemer som er for kraftige for nettbrett eller
liknende systemer man gjerne bruker til Twitter, f.eks. vil man kunne
gjøre en semi-automatisk analyse av en politisk debatt på TV, og så
kunne seerne samarbeide om å gjøre en faktasjekk i sanntid på det
politikerne sier. Resultatet kunne så distribueres gjennom sosiale
nettverk, og kunne også nå debattlederen, som kunne konfrontere
politikerne med opplysningene. Teknisk er dette mulig, men er det
mulig legalt?

Det er to kritiske spørsmål i så måte: Det ene er om det er mulig
innenfor plattformene som tilbys av f.eks. RiksTV, Get, osv. Det er
kanskje mulig, men dette er per idag meget primitive systemer, og hva
som skal inn der styres av kommersielle hensyn. I prinsippet er det
mulig at disse selskapene kunne støttet slik teknologi, men i praksis
ser det ikke ut til at dette skjer.

NoTube-prosjektet har
publisert\footnote{se \url{https://github.com/libbymiller/notube-jabber-mythtv}}
kode som bruker MythTV, som er et av de mer avanserte
mediesentersystemene på markedet. MythTV er fri programvare. For å
bruke MythTV må man kjøre kode som dekrypterer TV-signalet. Dette er
en nødvendighet for bruksscenariet over. Jeg har prøvd etter beste
evne å forstå om dette er lov eller ikke, men konklusjonen er at det
er det helt umulig å vite. 

Det er en lang rekke ting som gjør det umulig å vite. For det første
er dette til privat bruk, og dermed er det ikke opphavsrettslig
relevant. For det andre brukes kodekortet man har kjøpt og betalt for
på måten det er ment å brukes, så man baner seg ikke vei til andres
data, men på den annen side er det utvilsomt ment som et teknisk
beskyttelsessystem.

Hvis ikke det at det er ikke er opphavsrettslig relevant kan hjelpe,
så blir spørsmålet om systemet er effektivt sentralt. Det er her
forarbeidene i Ot. prp 46 2004-2005 går fullstendig feil. La oss
behandle følgende sitat:

\begin{quote}
Departementet antar at det i dette ikke ligger veldig strenge krav til
systemet, og at det f.eks. ikke kan kreves at det er så effektivt at
det nærmest er umulig å omgå. Samtidig kan systemet ikke være altfor
enkelt utformet, og dets formål må til en viss grad oppfylles. Det er
departementets oppfatning at systemer som nøytraliseres ved et
tusjstrøk på selve platen eller et tastetrykk ved innlesing i
datamaskin ikke vil oppfylle forslagets krav for beskyttelse, og at
systemer som er så enkle å omgå ikke vil være tekniske
beskyttelsessystemer i lovens forstand.
\end{quote}

Det er min oppfattning at definisjonen av et effektivt teknisk
beskyttelsessystem må være at det er umulig å omgå, og således bør det
heller ikke trenge lovbeskyttelse\footnote{Det kan bemerkes at det
virker som hele tanken med tekniske beskyttelsessystemer virker
fullstendig feilslått, slik at det slikt system ikke kan eksistere, og
at det er derfor man søker lovbeskyttelse.}. Etthvert system som skal
være mulig for en sluttbruker å bruke må være utformet slik at
brukeren oppfatter det som enkelt, dermed må det være en stor grad av
automatisering. Etthvert system som kan knekkes, kan også knekkes med
programvare som kan pakkes på en slik måte at en sluttbruker oppfatter
det som enkelt (det er også tilfelle med MythTV). Dermed vil enhver
teknisk beskyttelsesmekanisme være enten umulig eller enkel å
omgå. Departementets antagelse blir dermed feil. 

Enkelte jurister jeg har snakket med har med største letthet foreslått
at jeg bør ta saken til retten. Dette er en ekstremt navlebeskuende
holdning, konsekvensen av å ikke få medhold er faktisk ett års fengsel
og eller bøter i millionklassen, rett og slett personlig ruin. Det er
ufattelig for meg at det finnes jurister som ikke innser at kostnaden
er altfor høy, og at en minste tvil om legaliteten av et prosjekt vil
føre til at man rett og slett ikke gjør det. Slik stanser loven
teknologisk utvikling. Problemet er ikke nødvendigvis hva den sier,
men at det er formulert på en slik måte at det er helt uforståelig for
dem den retter seg mot\footnote{Forbrukermagasinet publiserte for noen
år siden en oversikt der de hadde spurt en rekke organisasjoners
jurister om hva de mente om hypotetiske problemstillinger liknende den
over, og den viste at juristene heller ikke har særlig
forståelse. Denne referansen er ikke lenger tilgjengelig på
nettet.}. Såvidt jeg kan forstå bryter dette også mot artikkel 29 i
menneskerettighetene, man må ha mulighet til å forstå en lov for å
ikke bryte den, og lovgiver må ta byrden med å gjøre dette klart
ettersom de ikke skal innføre begrensninger uten god grunn.


\paragraph{Etikk}

Min forståelse er at dette ikke er etiske spørsmål ettersom det ikke
er en reell interessekonflikt mellom skapere og mottakere av
verk. Disse må leve i et økosystem der de er gjensidig avhengige av
hverandre. 
