
I mars 2006 fikk jeg en mail der jeg ble bedt om å komme til London
for et jobbintervju påfølgende helg. I denne bransjen foregår selv
jobbintervjuer under taushetserklæring, men essensen var denne:
``Gründerne har nettopp solgt sitt forrige prosjekt for 2 milliarder
euro, og er klare for å revolusjonere TV-mediet. De skal gjøre det på
medieindustriens premisser, etter tidligere å ha vært i konflikt med
disse.'' Etter en prat ble jeg forespeilet at jeg kunne jobbe fra hvor
som helst, når som helst, og med den teknologien jeg selv syntes mest
interessant (og som jeg nå skriver mitt doktorgradsarbeid om), ``bare
si din pris''. Jeg avslo tilbudet.

I dette essayet skal jeg forklare hvorfor de premissene
medieindustrien legger er i vil ifrata befolkningen viktige friheter,
stanse teknologisk utvikling, gi et inntrykk i befolkningen at
vitenskapen er autoritær og ekskluderende, og dermed gjøre at all
vitenskapelig virksomhet faller i vanry. Videre har dette vide
implikasjoner for ytringsfrihet og demokrati. Men det verste er at
politikerne har allerede tatt lange skritt i denne retning, og dette
må reverseres.


\paragraph{Relevante menneskerettigheter og min forståelse}


\paragraph{Etikk}

Min forståelse er at dette ikke er etiske spørsmål ettersom det ikke
er en reell interessekonflikt mellom skapere og mottakere av
verk. Disse må leve i et økosystem der de er gjensidig avhengige av
hverandre. 
