
I mars 2006 fikk jeg en mail der jeg ble bedt om å komme til London
for et jobbintervju påfølgende helg. I denne bransjen foregår selv
jobbintervjuer under taushetserklæring, men essensen var denne:
``Gründerne har nettopp solgt sitt forrige prosjekt for 2 milliarder
euro, og er klare for å revolusjonere TV-mediet. De skal gjøre det på
medieindustriens premisser, etter tidligere å ha vært i konflikt med
disse.'' Etter en prat ble jeg forespeilet at jeg kunne jobbe fra hvor
som helst, når som helst, og med den teknologien jeg selv syntes mest
interessant (og som jeg nå skriver mitt doktorgradsarbeid om), ``bare
si din pris''. Jeg avslo tilbudet.

I dette essayet skal jeg forklare hvorfor de premissene
medieindustrien legger er i vil ifrata befolkningen viktige friheter,
stanse teknologisk utvikling, gi et inntrykk i befolkningen at
vitenskapen er autoritær og ekskluderende, og dermed gjøre at
vitenskapelig virksomhet faller i vanry. Videre har dette vide
implikasjoner for ytringsfrihet og demokrati. 

Dog ligger det i sakens natur at jeg som teknolog og naturviter kun
har overfladisk forståelse av de legale rammeverk som diskuteres, og
derfor vil jeg også beskrive tvil og problemer med bildet jeg tegner.

\paragraph{Relevante menneskerettigheter og min forståelse}

Den universelle menneskerettighetskonvensjonen\cite{udhr} er et
overordnet rammeverk for menneskerettigheter som er videre behandlet i
en rekke internasjonale eller overnasjonale avtaler. Dette er en
omfattende literatur jeg har lite grunnlag for å kjenne i detalj, og
jeg tar derfor utgangspunkt i konvensjonen.

I artikkel 29, 2. punkt kan man lese  
\begin{quote}
In the exercise of his rights and freedoms, everyone shall be subject
only to such limitations as are determined by law solely for the
purpose of securing due recognition and respect for the rights and
freedoms of others and of meeting the just requirements of morality,
public order and the general welfare in a democratic society.
\end{quote}

Det er min tolkning at begrensninger i individets frihet i form av lov
kun kan gjøres gjeldende der det er et klart og godt begrunnet behov
for dette. Videre må personene det gjelder faktisk ha mulighet til å
vite hva loven sier og hvordan man skal oppføre seg for å ikke bryte
den.

Artikkel 27 har to deler:
\begin{quote}
\begin{enumerate}
\item Everyone has the right freely to participate in the cultural
  life of the community, to enjoy the arts and to share in scientific
  advancement and its benefits.

\item Everyone has the right to the protection of the moral and
  material interests resulting from any scientific, literary or
  artistic production of which he is the author.
\end{enumerate}
\end{quote}

Opphavsrettslovgningen, i sine forskjellige former, skal gi den
beskyttelsen del 2 krever, mens del 1 har såvidt jeg vet ingen
eksplisitt beskyttelse. Dog begrunnes ofte sikring av forfatternes
materielle rettigheter med at det er det som gjør at befolkningen har
tilgang til verk.

\paragraph{Bevisbasert avgrensning}

Ettersom begrensninger i individets frihet må være klart begrunnet, og
et forbud mot eksemplarframstilling er en slik begrensning, så må vi
spørre om en del av utviklingen i senere år viser områder der slike
begrensninger er unødvendig.

Det er her naturlig å peke på Wikipedias suksess. Anekdotisk bevis
viser at Wikipedia kan vurderes på linje med etablerte leksika, se
omtale~i~\cite{naturewiki}. Hvis man godtar at Wikipedia er bra nok,
så må man så spørre seg om dette kunne vært oppnådd uten de
begrensninger åndsverksloven gir. Da må vi notere oss at Wikipedia
også har lisensbetingelser, men disse regulerer kun forhold som er
relatert til dagens opphavsrettsregime, de ville ikke være nødvendige
i et regime der leksika ikke hadde vern, ei heller regulerer de
eksemplarframstilling. Dermed bør Wikipedia tjene som et eksempel på
at begrensninger i befolkningens frihet når det gjelder leksika
\emph{ikke} er nødvendig.

Vi ser det samme i mange andre felter, både innenfor informasjon, data
og programvare. Man kan idag fint leve uten programvare som i et
regime der programvare ikke var vernet likevel ville eksistert i sin
nåværende form. Dette er programvare som gjerne benevnes som ``fri
programvare'' . Jeg bruker selv nesten utelukkende fri programvare, og
utgir selv det jeg skriver som fri programvare, og man kan finne alt
fra meget enkle programmer til meget avanserte systemer for
resonnering. Blant web-servere så er det to slike systemer som
konkurrerer i popularitet med Microsofts begrensende
tilbud\cite{netcraft}. Det er min oppfatning at dette betyr at
programvare i hovedsak ikke bør kunne underlegges begrensninger.

Dog ser jeg at det kan finnes unntak fra denne regelen. Jeg ser det
som utvilsomt at det er enklere å tjene penger på programvare som er
underlagt begrensninger basert på egen erfaring. Menneskerettighetene
sier ikke at det skal være enkelt å tjene penger, men det kan tenkes
at det blir så vanskelig å få til en investering i f.eks. veldig
sektor-spesifik programvare at ingen vil ta seg råd til å utvikle
den. Dette vil dermed gjøre at man ikke får til teknologisk utvikling.

Vi må videre spørre, gitt suksessen til Wikipedia og til fri
programvare, hvilke andre områder er det det er illegitimt å begrense
befolkningens frihet? Hvilke andre områder vil man kunne ha samme eller
forbedret kunstnerisk og vitenskapelig framgang uten slike begrensninger?

Jeg foreslår ikke en radikal avskaffelse av opphavsretten,
hovedpoenget er at enhver begrensning må begrunnes, og at man må være
svært forsiktig med å legge gjennomgripende begrensninger. Ettersom
enhver begrensning er en inngripen i befolkningens frihet, er det
naturlig at det stilles beviskrav til den som foreslår en begrensning.  

\paragraph{Avgrensning av inngripende tekniske tiltak}


\paragraph{Etikk}

Min forståelse er at dette ikke er etiske spørsmål ettersom det ikke
er en reell interessekonflikt mellom skapere og mottakere av
verk. Disse må leve i et økosystem der de er gjensidig avhengige av
hverandre. 
