\documentclass{llncs}

\begin{document}
\title{Sharing statistics for SPARQL Federation optimization, with
  emphasis on benchmark quality}
\author{Kjetil Kjernsmo}
\institute{Department of Informatics,
Postboks 1080 Blindern,
0316 Oslo, Norway}

\email{kjekje@ifi.uio.no}

\maketitle

\begin{abstract}


\end{abstract}

\section{Introduction}

Query federation has been an active field for some time, but has until
the advent of the Semantic Web not been used in a highly diverse set
of endpoints, commonly they have been under the control of a single
institution. RDF has a triple-based data model for the Semantic Web,
and SPARQL is a standardized query language to query such data.

Query federation with SPARQL has attracted much attention
from industry and academia alike, and four implementations of basic
query federation were submitted to the SPARQL 1.1 Working Group as
input for the forthcoming work. The basic query federation feature was
supported by a large number of group members, and the latest working
draft of the proposed standard was published on June 1st 2010 and is
expected to enter a Last Call review period shortly. % TODO: Check
                                % status

While the basic feature set of the proposed standard can enable users
to create federated queries, it is not of great use as it requires
extensive prior knowledge of both the data to be queried and
performance characteristics of the involved query engines. Without
this knowledge, the overall performance is insufficient for any
practical applications.

I intend to investigate possible remedies to this problem by using
statistical techniques. 

Since the main objective of the proposed work is to create systems
that have sufficient performance for practical applications, it is of
paramount importance to have methodology that can falsify a
conjecture about an implementation's performance.

In a federated regime, any query is likely to execute queries on
several different SPARQL implementations. Many of those are free
software so the algorithms they use could in principle be analyzed,
but in practice, this would be very time-consuming. Some
implementations may be undocumented or even trade secrets, thus
precluding any such analysis. It is therefore my intention to treat
the individual SPARQL implementations as ``black boxes'' and
exclusively evaluate the performance characteristics by empirical
means, aka ``benchmarking''.

However, I have found the current state of the art in benchmarking
lacking in its use of statistics, which motivates two distincts
directions of work: \emph{Statistical experimental planning and execution in
benchmarking} and \emph{statistical methods to optimize SPARQL queries
in a federated regime}.

\section{Problem Definition}

\section{State of the Art}


\section{Proposed Approach and Methodology}

\section{Conclusion}


\end{document}
