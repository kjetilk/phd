\documentclass{llncs}
%\documentclass{article}
\usepackage{cite}

\title{Practical implications of HATEOAS on a bridge between LOD and SPARQL}
\titlerunning{HATEOAS and a bridge between LOD and SPARQL}
\author{Kjetil Kjernsmo\inst{1}}
\institute{Department of Informatics,
Postboks 1080 Blindern,
0316 Oslo, Norway
\email{kjekje@ifi.uio.no}}


\begin{document}

\maketitle



\begin{abstract}
This paper makes two points: 1) SPARQL is a necessity % TODO: relax 
for non-trivial navigation in Linked Data, 
2) The community needs to pay attention to
the HATEOAS constraint of REST and its practical implications for
making a bridge between LOD and SPARQL possible.


\end{abstract}

\section*{BRAINDUMP STATUS}

\section{Introduction}

Some bloggers in the developer community, see e.g.~\cite{sunsetonmvc},
is currently predicting a paradigm shift in common practices, prompted
by the trend towards richer clients. This has caused a departure from
the time-tested Model-View-Controller pattern that has been the
dominant paradigm in web development for more than a decade.

Semantic Web services is likely to be an important source of data for
future applications, but for foreseeable future just one of many. A
key promise of Linked Open Data (LOD) in particular, and the Semantic
Web in general is the ability to integrate many data sources
easily. However, this integration requires links and it requires that
the resulting graph can be traversed by reasonable means. While links
are quite abundant across the LOD cloud, this paper will argue that
key links are missing to make it possible to traverse the resulting
graph. Moreover, we will argue that this shortcoming is due to that
the community does not adequately take into account the constraint
known as ``Hypermedia as the Engine of Application State'' (commonly
abbreviated HATEOAS) from the REST architectural style, see
\cite{Fielding_2000_Architectural-Styles}~Chapter~5.

The HATEOAS constraint requires that the application can navigate from
one resources to another using the hypermedia links in the present
resource \emph{only}, it should not require any out-of-band
information to do so. Thus, a RESTful API should not require prior
knowledge of a URI structure, but one may define the API in terms of
URI structure as long as the same information can be gleened from the
resources. Nor does it require that all resources can be reached from
any other resource. As such, the constraint is not very strict, one
could argue that as long as there are some links, the API satisfies
the HATEOAS constraint.

Since RDF is built on URIs, many of which can be dereferenced to
obtain further links, it lends itself well to RESTful APIs, but it is
debateable whether an API can be said to be RESTful if the links
doesn't enable any useful interactions. Therefore, whether the HATEOAS
constraint is satisfied must be judged on the basis of the practical
applications the API enables.

The LOD Cloud consists of well-linked datasets and to find simple
answers by simply dereferencing URIs, parse the returned content and
address it by conventional methods provided by the programming
language of choice, and this pattern has proved very useful. However,
for slightly more complex applications, one has to traverse a
relatively large graph to find the data one requires. In most cases,
one would need to match the graph with a pattern, which is essentially
what the SPARQL query language does. In the Linked Data Design
Issue\footnote{\url{http://www.w3.org/DesignIssues/LinkedData}} Tim
Berners-Lee notes ``To make the data be effectively linked, someone
who only has the URI of something must be able to find their way the
SPARQL endpoint.'', but this is generally not possible today without
requiring out-of-band information, which implies that non-REST
interactions are required. To remedy this situation, we would need to
add more links.

Note that the bridge between LOD and SPARQL discussed in this paper is
different from the bridge provided by e.g. Pubby or RDF::LinkedData
and from the Linked Data API that is written to simplify use of linked
data by web developers\footnote{http://code.google.com/p/linked-data-api/}.

\section{Non-trivial graph traversal}

\section{Lacking links in standards}

The SPARQL 1.1 Graph Store HTTP
Protocol\footnote{\url{http://www.w3.org/TR/2011/WD-sparql11-http-rdf-update-20110512/}},
currently with a Last Call status, claims to define a protocol in the
REST architectural style. It is not clear how it satisfies the
constraints of REST. On one hand, the strong ties to HTTP is overly
restrictive from a REST point of view, on the other, the specification
contains no mention today of how hypermedia (i.e. RDF in this case)
will drive interaction, which is the critical part of a using the REST
architectural style. Moreover, it defines out-of-band URI keywords
(\texttt{graph}). As mentioned in the introduction, this is not
necessarily a violation of architectural constraints, but the
specification has no mention of how applications can interact to
achieve the same tasks as it could with this out-of-band information.

This represents a lost opportunity to consider the practical
implications of HATEOAS to come up with a clearly RESTful specification.

% TODO: more than one URI for graph traversal


\section{Proposed solutions}

Prefixes are in the following omitted for brevity.

\subsection{Simple hypermedia vocabulary}

To help drive interaction through hypermedia, the RDF information
resources has to contain information about the URIs that can be used
to perform operations other than pure reads. To this end, we suggest a
simple vocabulary that can be used to supply some statements with all
resources that the client may be allowed to manipulate.

In keeping with the REST philosophy that the representation should be
independent of the protocol that is used to transfer it, the
vocabulary is independent of HTTP, but the explanatory comments uses
HTTP examples, as this makes the discussion clearer.

\begin{verbatim}
hm:operations a rdf:Property ;
              rdfs:range foaf:Document ;
              rdfs:domain hm:Operations .
\end{verbatim}
\texttt{hm:operations} link the resource we want to manipulate,
typically the current information resource to the operations we may be
allowed to perform on it.


In all the following examples, the object of the statement is the
Request-URI that the server suggests is used to manipulate the resource.
\begin{verbatim}
hm:mergeInto a rdf:Property ;
             rdfs:range hm:Operations ;
             rdfs:domain foaf:Document .
\end{verbatim}
\texttt{hm:mergeInto} is used to indicate that a HTTP POST operation
will do an RDF merge of the request body with the content at the
Request-URI.

\begin{verbatim}
hm:serverCreate a rdf:Property ;
                rdfs:range hm:Operations ;
                rdfs:domain foaf:Document .
\end{verbatim}
\texttt{hm:serverCreate} is used to indicate that a HTTP POST
operation to the Request-URI will create a new resource with a
server-supplied URI.

\begin{verbatim}
hm:delete a rdf:Property ;
          rdfs:range hm:Operations ;
          rdfs:domain foaf:Document .
\end{verbatim}
\texttt{hm:delete} is used to indicate that the resource should be
deleted when issuing a HTTP DELETE operation to the Request-URI. 

\begin{verbatim}
hm:put a rdf:Property ;
       rdfs:range hm:Operations ;
       rdfs:domain foaf:Document .
\end{verbatim}
\texttt{hm:put} is used to indicate that the resource should be
created using the request body or updated with it if it already exists
when issuing a HTTP PUT operation to the Request-URI.

\begin{verbatim}
hm:Operations a rdfs:Class .
\end{verbatim}
The class of possible operations.

To manipulate default graphs, properties \texttt{hm:mergeIntoDefault},
\texttt{hm:deleteDefault} and \texttt{hm:putDefault} are defined
similarly.

In most cases, the current information resource will be the subject of
the \texttt{hm:operations} \strong{but also} the object of the allowed
operations. A typical information resource will supply the following
triples (given a suitable base URI):

\begin{verbatim}
<> hm:operations
   [ a hm:Operations ;
     hm:mergeInto <> ;
     hm:delete <> ;
     hm:put <> ;
     hm:serverCreate <../> .
   ] .
\end{verbatim}
This represents the common case when you GET a resource, then update
it with PUT or DELETE or append triples with POST.


\subsection{Finding the way with VoID}

\begin{verbatim}
<> void:inDataset [ void:sparqlEndpoint </sparql> . ] .
\end{verbatim}

\section{Questions for discussion}

\begin{enumerate}
\item Tim Berners-Lee suggests in the Linked Data Design
  Issue\footnote{\url{http://www.w3.org/DesignIssues/LinkedData}} the
  use of ``Minimum Spanning Graph'', what other triples would humans
  and machines find useful?

\item What can we do to add more H Factors\footnote{see
  \url{http://amundsen.com/hypermedia/hfactor/}} to RDF?

\end{enumerate}


\bibliographystyle{plain-csmin}
%\bibliographystyle{jbact}
%\bibliographystyle{splncs03}
\bibliography{federation,webarch,webframeworks}

\end{document}
