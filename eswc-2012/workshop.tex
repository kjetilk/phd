\documentclass{llncs}
%\documentclass{article}
\usepackage{cite}

\title{Practical implications of HATEOAS on a bridge between LOD and SPARQL}
\titlerunning{HATEOAS and a bridge between LOD and SPARQL}
\author{Kjetil Kjernsmo\inst{1}}
\institute{Department of Informatics,
Postboks 1080 Blindern,
0316 Oslo, Norway
\email{kjekje@ifi.uio.no}}


\begin{document}

\maketitle



\begin{abstract}
This paper makes two points: 1) SPARQL is a necessity for non-trivial
navigation in Linked Data, 2) The community needs to pay attention to
the HATEOAS constraint of REST and its practical implications for
making a bridge between LOD and SPARQL possible.


\end{abstract}

\section{Introduction}

Some bloggers in the developer community, see e.g.~\cite{sunsetonmvc},
is currently predicting a paradigm shift in common practices, prompted
by the trend towards richer clients. This has caused a departure from
the time-tested Model-View-Controller pattern that has been the
dominant paradigm in web development for more than a decade.

Semantic Web services is likely to be an important source of data for
future applications, but for foreseeable future just one of many. A
key promise of Linked Open Data (LOD) in particular, and the Semantic
Web in general is the ability to integrate many data sources
easily. However, this integration requires links and it requires that
the resulting graph can be traversed by reasonable means. While links
are quite abundant across the LOD cloud, this paper will argue that
key links are missing to make it possible to traverse the resulting
graph. Moreover, we will argue that this shortcoming is due to that
the community does not adequately take into account the constraint
known as ``Hypermedia As The Engine Of Application State'' (commonly
abbreviated HATEOAS) from the REST architectural style, see
\cite{fielding}~Chapter~5.

The HATEOAS constraint requires that the application can navigate from
one resources to another using the hypermedia links in the present
resource \emph{only}, it should not require any out-of-band
information to do so. Thus, a RESTful API should not require prior
knowledge of a URI structure, but it does not forbid it as long as the
same information can be gleened from the resources. Nor does it
require that all resources can be reached from any other resource. As
such, the constraint is not very strict, one could argue that as long
as there are some links, the API satisfies the HATEOAS constraint.cc
% TODO: It is practical



%\bibliographystyle{plain-csmin}
%\bibliographystyle{jbact}
%\bibliographystyle{splncs03}
%\bibliography{selectivity,federation,benchmarks,optimization}

\end{document}
