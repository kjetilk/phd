\documentclass{llncs}
%\documentclass{article}
\usepackage{cite}

\title{The necessity of hypermedia RDF and an approach to achieve it}
\titlerunning{A hypermedia RDF approach}
\author{Kjetil Kjernsmo\inst{1}}
\institute{Department of Informatics,
Postboks 1080 Blindern,
0316 Oslo, Norway
\email{kjekje@ifi.uio.no}}


\begin{document}

\maketitle



\begin{abstract}
This paper will give an overview of the practical implications of the
HATEOAS constraint of the REST architectural style, and in that light argue why
hypermedia RDF is a practical necessity. We will then sketch a
vocabulary for hypermedia RDF using Mike Amundsen's H~Factor
classification as motivator. Finally, we will briefly argue that
SPARQL is important when making non-trivial traversals of Linked Data
graphs, and see how to a bridge between Linked Data and SPARQL may be
created with hypermedia RDF.
\end{abstract}

\section*{BRAINDUMP STATUS}

\section{Introduction}

%<> hm:can-be hm:put-to ;
% hm:can-be hm:deleted .

Mike Amundsen defines hypermedia types\footnote{\url{http://amundsen.com/hypermedia/}} as 
\begin{quote}
Hypermedia Types are MIME media types that contain native
hyper-linking semantics that induce application flow. For example,
HTML is a hypermedia type; XML is not.
\end{quote}
Furthermore, he defines a classification scheme called H~Factor as ``a
measurement of the level of hypermedia support and sophistication of a
media-type.'' The REST & WOA Wiki defines ``the Hypermedia
Scale''\footnote{\url{http://restpatterns.org/Articles/The_Hypermedia_Scale}},
where the categorization is based on capabilities for create, read,
update and delete operations.  Standing on it's own, RDF is a
hypermedia type, but only at the \textsf{LO} and \textsf{CL} levels
on, and just an R~Type (read) on the Hypermedia Scale. We aim at improving
this situation.
