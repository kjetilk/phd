\documentclass[12pt, draft]{article}
\usepackage{cite}

\newcommand{\rdfterm}[1]{\texttt{#1}}
\newcommand{\todo}[1]{\ensuremath{^{\textrm{\tiny{TODO}\normalsize}}}\footnote{\textbf{TODO:}~#1}}

\title{Conditional selectivity for Hexastores}

\author{Kjetil Kjernsmo}

\begin{document}
\maketitle

\section*{Status of this document}

This is currently at braindump status. Do not distribute without the
permission of the author.

\begin{abstract}
  This paper explores selectivity estimation for single triple
  patterns in SPARQL queries. In this paper we see how the common
  assumption of independence of the selectivity of RDF terms can be
  abandoned. We shall also find it instructive to explore how
  selectivity can be determined exactly when a SPARQL query is
  executed on a Hexastore triple store
  \cite{Weiss:2008:HSI:1453856.1453965}. Finally, we will suggest
  further directions for selectivity estimation in the conditional
  regime.

\end{abstract}

\section{Introduction}

Several authors have studied the use of selectivity estimation for
optimizing SPARQL query execution plans, both in the context of Basic
Graph Patterns \cite{Stocker:2008:SBG:1367497.1367578} and in more
general cases. \todo{more references. OptARQ? Some go into conditional
regimes, actually.}

The above authors assume that the selectivity of RDF terms are
independent, admit that this assumption is easily violated and
continue to use a conditional estimate for the object node.

It is easy to point out cases where the independence assumption is
unlikely to hold: If a predicate is \rdfterm{rdf:type}, then only
classes are likely objects, and their frequency  would likely be
different from e.g. literals.

\section{Conditional selectivity}

Previous authors have taken selectivity as analogous to the
probability of matching a SPARQL term (RDF Term or variable) with an
RDF term in the data or a SPARQL triple pattern with a triple in the
data.  For a triple, $<S, P, O>$ the selectivity is the intersection
of the selectivity of each term.

From elementary statistics, we know that the intersection probability
can be written as
\begin{eqnarray}
Pr(A_1 \cap A_2  \cap \ldots \cap A_n) &=& 
  Pr(A_n | A_1 \cap A_2  \cap \ldots \cap A_{n-1}) \nonumber\\
  &&\cdot Pr(A_{n-1} | A_1 \cap A_2  \cap \ldots \cap A_{n-2}) \cdot \ldots \nonumber\\
  && \cdot Pr(A_2 | A_1) \cdot Pr(A_1) ,
\end{eqnarray}
where the $|$ symbol is read as ``given'', e.g. ``given a predicate
\rdfterm{rdf:type} what is the probability that the object is \rdfterm{foaf:Person}?''

Analogously\todo{anybody good at arguing this mathematically?}, the
conditional selectivity for a triple pattern $T$ can be written as 
\begin{equation}\label{eq:sel}
sel(T) = sel(S \cap P \cap O) = sel(O | S \cap P) \cdot sel(P | S)
\cdot sel(S) . 
\end{equation}
We note that intersection is commutative, thus similar
relations can be shown for each permutation of terms.

Finally, note that for variables, the selectivity will always be
unity, whether conditional or independent, since it will match all
possible RDF terms.

\subsection{Example}\label{sec:example}

Consider the following RDF graph, expressed in Turtle with base URI
and namespace declarations omitted for brevity:

\begin{verbatim}
</alice> a foaf:Person ;
         foaf:firstName "Alice" ;
         rel:spouseOf </bob> .
</bob>   a foaf:Person ;
         foaf:firstName "Bob" ;
         foaf:based_near </place> .
</place> rdfs:label "Our House" .
\end{verbatim}

A very simple example is to find the selectivity for the triple pattern
\begin{verbatim}
?person a foaf:Person .
\end{verbatim}
We can easily see that the correct answer is $sel(T) = 2/7$ since two
out of the 7 triples match the pattern. To see how Eq.~\ref{eq:sel} is
used, we note that $sel(S) = 1$ since it is a variable. $sel(P | S) =
2/7$ since the predicate matches the two occurences of
\rdfterm{rdf:type} and the subject variable doesn't influence
this. Finally, $sel(O | S \cap P) = 1$ since the only two occurences
of \rdfterm{foaf:Person} occurs in the same triples as
\rdfterm{rdf:type}. Thus, we arrive at the expected result.

In a slightly harder example:
\begin{verbatim}
?person foaf:firstName "Bob" .
\end{verbatim}
We still have $sel(S) = 1$ as above, and there are still two
predicates that match, so $sel(P | S) = 2/7$, but only half of those
match the literal, so $sel(O | S \cap P) = 1/2$, giving us the
expected $sel(T) = 1/7$.

\section{Exact selectivity for Hexastore}

The Hexastore was first described in
\cite{Weiss:2008:HSI:1453856.1453965}, which sought to exploit a key
strength of the RDF triple model by using six-way indices for
efficient lookup.

A Hexastore contains the six indices \textsf{spo}, \textsf{pso},
\textsf{osp}, \textsf{sop}, \textsf{pos} and \textsf{ops}. Each index
has the following structure, taking \textsf{spo} as an example: ``A
subject key $s_i$ is associated to a sorted vector of $n_i$ property
keys, $\{p_1^i , p_2^i , \ldots , p^i_{n_i} \}$. Each property key $p_j^i$
is, in its turn, linked to an associated sorted list of $k_{i,j}$
object keys.''\cite{Weiss:2008:HSI:1453856.1453965}, section~4.1.

Let us return to the example in section~\ref{sec:example}: We have
$s_1 = \texttt{key-for-alice}$ and $i=2$. Further, $p_1^1 =
\texttt{key-for-foaf:firstName}$, $p_2^1 = \texttt{key-for-rdf:type}$ and $p_3^1 =
\texttt{key-for-rel:spouseOf}$ and $n_1 = 3$. For each predicate, there is only
one object, so $k_{1,1} = 1$ and $o_{1_{1,1_1}}^{1,1_1} = \texttt{key-for-Alice-literal}$.

The pattern that emerges is that since the index for the predicate is
already dependent on the match for the subject, the conditional
selectivity depends only on the number of matches of a term and the
total number of keys in the index. Thus, the selectivity can be
computed \emph{exactly} based on the data only. The selectivity of a
query then depends only on what indices are used. 

Formalizing this notion, we first define a function $cm(x)$ which
returns the number of matches of a given key $x$\todo{argue the
  term-key-connection}. 

Then, 
\begin{equation}
sel(S = s_i) = \frac{cm(s_i)}{i} ,
\end{equation}
\begin{equation}
sel(P = p_{n_i}^i|S = s_i) = \frac{cm(p_{n_i}^i)}{n_i}
\end{equation}
and
\begin{equation}
sel(O = o_{k_{i,n_i}}^{i,n_i} |S = s_i \cap P = p_{n_i}^i) =
\frac{cm(o_{k_{i,n_i}}^{i,n_i})}{k_{i,n_i}} .
\end{equation}
\todo{surely needs more work}

With this, it should be possible to store the selectivity along with
the indices in a Hexastore.


\section{Further work on conditional selectivity}

\todo{this section}

\bibliographystyle{plain}
\bibliography{selectivity}



\end{document}
