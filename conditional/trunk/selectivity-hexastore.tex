\documentclass[12pt]{article}
\usepackage{cite}

\newcommand{\rdfnode}[1]{\texttt{#1}}

\title{Conditional selectivity for Hexastores}

\author{Kjetil Kjernsmo}

\begin{document}
\maketitle

\begin{abstract}
In this paper we see how the common assumption of independence of the
selectivity of RDF nodes can be abandoned. We shall also find it
instructive to explore how selectivity can be determined exactly when
a SPARQL query is executed on a Hexastore triple store
\cite{Weiss:2008:HSI:1453856.1453965}. Finally, we will suggest
further directions for selectivity estimation in the conditional
regime.

\end{abstract}

\section{Introduction}

Several authors have studied the use of selectivity estimation for
optimizing SPARQL query execution plans, both in the context of Basic
Graph Patterns \cite{Stocker:2008:SBG:1367497.1367578} and in more
general cases. % TODO more references. OptARQ?

The above authors assume that the selectivity of RDF nodes are
independent, admit that this assumption is easily violated and
continue to use a conditional estimate for the object node.

It is easy to point out cases where the independence assumption is
unlikely to hold: If a predicate is \rdfnode{rdf:type}, then only
classes are likely objects, and their frequency  would likely be
different from e.g. literals.

\section{Conditional selectivity}

Previous authors have taken selectivity as analogous to the
probability of matching a SPARQL node or triple with a node or triple
in the data. For a triple, $<S, P, O>$ the selectivity is the
intersection of the selectivity of each node.

From elementary statistics, we know that the intersection probability
can be written as
\begin{equation}
Pr(A_1 \cap A_2  \cap \ldots \cap A_n) = 
  Pr(A_n | A_1 \cap A_2  \cap \ldots \cap A_{n-1}) \\
  \cdot Pr(A_{n-1} | A_1 \cap A_2  \cap \ldots \cap A_{n-2}) 
  \cdot \ldots \cdot Pr(A_2 | A_1) \cdot Pr(A_1) ,
\end{equation}
where the $|$ symbol is read as ``given'', e.g. ``given a predicate
\rdfnode{rdf:type} what is the probability that the object is \rdfnode{foaf:Person}?''

Analogously, the conditional selectivity for a triple $T$ can be written as 
\begin{equation}
sel(T) = sel(S \cap P \cap O) = sel(O | S \cap P) \cdot sel(P | S)
\cdot sel(S) . 
\end{equation}


\bibliographystyle{plain}
\bibliography{selectivity}



\end{document}
