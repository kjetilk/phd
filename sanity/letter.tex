\documentclass{article}
\usepackage[english,british]{babel}

\usepackage[utf8]{inputenc}
\usepackage{cite}
\usepackage{verbatim}
\usepackage{graphicx}
\usepackage{hyperref}
\usepackage{appendix}

\newcommand{\rdfterm}[1]{\texttt{#1}}
\newcommand{\pvalue}{\textit{p}-value\ }
\newcommand{\httph}[1]{\texttt{#1}}
\newcommand{\todo}[1]{\ensuremath{^{\textrm{\tiny{TODO}\normalsize}}}\footnote{\textbf{TODO:}~#1}}

\title{Letter concerning changes in ``A survey of HTTP caching implementations on the open Semantic Web''}
\author{Kjetil Kjernsmo}

%\subtitle{---Unpublished Working Draft, do not circulate---}


\begin{document}
\maketitle

I have as requested made extensive changes to the submitted
paper. First, the appendices were moved to a Technical Report, due to
be published by the University of Oslo shortly as
\cite{kjernsmo_add_survey_2015}. This report will also include some
extra material, particularly statistical methods, that were removed
from the paper before submission.

\section{Meta-review items}

\begin{enumerate}

\item We include Table~2, which lists the \httph{Server} headers for
  hosts that enabled a freshness lifetime to be computed for
  \emph{all} requests. It makes the conclusion that not much can be
  found from these headers clear, and shows how they should look. For
  the contingency table test, the reader is referred to
  \cite{kjernsmo_add_survey_2015}.

\item Number of occurrences is now explained in the caption. The
  percentage numbers have been hacked to display well.

\item This comment seems to pertain to different aspects of the paper:
  First, I acknowledge that I quote many numbers in the text, outside
  of the descriptive statistics, and I understand that this may seem
  disorganised, and that the desire may emerge to see them nicely
  tabulated. They don't lend themselves well to tabular display, since
  they come from quite different parts of the analysis. For example, I
  don't think it would serve paper well if the figures of the
  \httph{Pragma} headers would be in a table with the
  \httph{Cache-Control} headers, the discussion is too
  different. Also, I note that the majority of the reviewers rated the
  paper well with respect to clarity. The other aspect of the paper
  seems related to coverage. To answer that, I have noted that the
  relative sizes of the databases are not important. See also the next
  item.

\item I have made a separate section entitled ``Challenges to
  validity'' to address some of these concerns. In retrospect, using
  LODstats for a list of vocabularies would probably have spared me
  much trouble, but should not challenge the validity much. However, I
  acknowledge that I have not quite understood what
  \cite{stateoflod2014} did, as the LOD cloud is now quite a lot
  larger than the BTC2014 dataset, but the website quoted in the
  review has only crawled 188 MTriples, which isn't at all large.

\item First, it is clearly out of the scope of the paper to discuss
  how to cache and get a performance improvement. As this is the focus
  of my current work, I took the opportunity to talk about it dearly
  anyway, and extended the future work section with a discussion. I
  note that the reviewer had changed their review to ask us to
  ``acknowledge the complexity to cache SPARQL'', which I think I had
  done in the related work section where I wrote that ``In some of the
  related work, it is shown that caching does not necessarily give
  tangible benefits.'', but I didn't at that time find space to
  elaborate. With space freed by the technical report, I have now done
  so anyway.

\item A very brief comment in ``Future work'', as doing this as a
  major undertaking. Ultimately, it can probably only be done by an
  extensive survey using questionnaires.

\item I acknowledge the error, and reran the spider recording
  \httph{Surrogate-Capability}, which is the correct header, but that
  were not observed either.

\item First, I changed ``RFC7234, Section~4.2.2'' to ``Section~4.2.2
  in RFC7234'', which I hope should clarify that it pertains to the
  standard. Second, since it is my agenda to spread knowledge about
  foundational Web technologies, I would like to follow this
  suggestion. However, to satisfy all the other requirements, I found
  this hard to prioritise, after all this is about the HTTP standard,
  a standard that is fundamental to all practical contributions to the
  Semantic Web. Since review 5 requested further elaboration on how
  the revalidation process works, I detailed that. I also reintroduced
  the table of all recorded headers, that previously was in the
  appendix.

\item See Section~\ref{rev6}.

\item I have made another pass of the paper, and also sent it to a
  professional English proofreader, but don't know if it will fit in
  their schedule to review it before the deadline.

\end{enumerate}

\section{Reviewer 6's comments}\label{rev6}

\begin{description}
\item [Contribution list] I have rewritten two paragraphs, essentially
  keeping the objectives, but detailing how meeting those objectives
  realise the contributions.
\item [Preliminaries and Fundamentals] I have, as noted about, further
  expanded on the RFCs, and put that in a separate subsection under
  Introduction, as I think the text flows better with that
  arrangement. I also included the table as per their suggestion.
\item [Methodology] I identified the the source of the correlation
  that reviewer mentioned as due to a single host having many SPARQL
  endpoints configured similarly and noted that.
\item [Data] I have included a URL to my homepage, where I will make
  an archive available shortly with data and code (I have to admit
  that the code isn't all that pretty, I didn't intend the survey to
  become as extensive as it that, that's why I didn't just release it
  immediately).
\item [Analysis1] I'm not quite sure what they are getting at, but
  included a paragraph just after the section headline.
\item [Analysis2] See Section~\ref{histogramstuff}.
\item [Discussion] This is really about an extensive restructuring of
  the paper, see Section~\ref{restructure} for my take.
\end{description}

\subsection{Histogram breaks}\label{histogramstuff}

The reviewer remarks that the diagrams do not follow clean
mathematical mechanics. This is a very interesting comment, and one
that I have devoted much thought to both while writing the paper, and
after receiving the reviews. 

First, it is important to note that histograms, mosaic plots,
boxplots, density plots are all ways to visualise summarised
distributions of data. As summarisations, they can be seriously
misleading, even deceiving, if done wrong. Moreover, I acknowledge
that violating ``mathematical mechanics'' in histograms is a
frequently seen technique to ``lie with statistics''. 

However, the key goal of the paper is to communicate some essential
characteristics, and I had already before the paper was submitted used
a logarithmic scale in those plots, like the reviewer suggested. They
did not communicate well, since we do not live on a scale of seconds
when we talk beyond a few minutes.

I had not, however used both axis, as the reviewer suggested, so I
have now done that, see Figure~\ref{fig:log}.

\begin{figure}[h!]
  \centerline{%
    \includegraphics[width=.9\textwidth]{logandtime.pdf}}
  \caption{Histogram counting all standards-compliant freshness
    lifetimes found. The data is presented on a logarithmic scale,
    with the lower x-axis giving the lifetime in seconds and the upper
    x-axis dividing into ``off'', and then times in usual
    calendar. Sans polishing.}
  \label{fig:log}
\end{figure}

I still opted not to polish and include this into the paper, as
including the axis on top didn't solve the fundamental problem of
communication: One doesn't have any everyday feeling for how long
100~000 seconds is, that is seldomly of interest, rather, one is more
likely to ask how many months a cache may be valid, and in
Figure~\ref{fig:log} that question cannot be answered. 

Further, if one is interested in the freshness for a resource at that
level of detail, it is more likely that the primary interest is actual
change frequency, a topic that is already better covered by DyLDO \cite{dyldo2}.

The only advocacy I derive from these graphs is the suggestion that
cache prohibition is misguided, which is something that the violation
of mathematical mechanics doesn't change. Other than that, the
intention here is purely descriptive, and that is better served with
the current breaks.

\subsection{Restructure}\label{restructure} 

Review 6 asked for a major restructure of the paper. I have noted that
most of the other reviewers found that the the clarity was good, also
across expertise levels: Both reviewer 4, who indicated expert
confidence and reviewer 1, who indicated confidence 3, rather the
paper with ``very clear''. Reviewer 5 also write that the paper is
very well presented even though they give it a 3 rating on
clarity. I'm therefore wary about making changes that may ruin the
clarity for the majority, to satisfy what seems to be a minority.

I have therefore rather tried to understand what reviewer 6 finds
problematic about it than take their suggestions literally. I have
therefore done little to the Analysis section, apart from providing an
initial overview. 

Clearly, reviewer 6 as found the diagrams as not conforming to usual
form, and they are like reviewer 2, somewhat confused by all the
numbers that appear in the text, which indeed is somewhat problematic
but not easily improved.

However, I think the main concern of reviewer 6 is that they put an
emphasis of the recommendations that I didn't intend. The paper is
first and foremost meant descriptive, and that's the reason the
Recommendations is just a tiny section at the end. The only
recommendation that is derived directly from the paper is the one on
cache prohibitions. The rest of the recommendations are based in part
of that I've been working with this for a decade.

Therefore, I have rather than restructured the paper, clarified this
in the Recommendations, but also in the contributions, and also
changed wording here and there. I hope this makes it possible to read
the paper with different eyes, and that it will be clearer to
everyone.

%\bibliographystyle{plain}
\bibliographystyle{abbrv}
%\bibliographystyle{jbact}
%\bibliographystyle{splncs03}
\bibliography{egne,webarch,data,rfc,hypermedia,dynamicity,stat1,specs,vocabs,similarity,optimization}



\end{document}
