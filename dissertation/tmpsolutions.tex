\section{Possible directions of work}

\emph{This section is a temporary discussion area}

We need to discuss the future strategy for this work. I think the main
discussion point is the tension between elaborately develop theory
that then can be applied to derive publishable results, vs. the
practical time constraints of Ph.D. work. Another issue is that I'd
like to be mindful of the realities of adoption in standards and
adoption of standards.

\subsection{VoID extensions}

The SPLENDID paper provides a pretty clear path forward to really
solve the \langcase: Certain statistics could be amended to a new
release of VoID for cases where it very clearly would have impact. It
would be up to the data management software to identify those case
based on the work I publish.

There are two basic classes of solutions to this problem: One is to
allow further nesting of dataset partitions. VoID currently allow
class and property partitions, where the class partitions are the most
important in SPLENDID, property partitions is the most interesting for
conditional statistics. Within property partitions, one could imagine
a nested object partition, and the \langcase{} could be solved
conclusively by specifying the number of triples containing each
language. If we did the same thing with subject partitions, it would
be a very good solution to this and similar problems. However, it
would be a rather verbose solution. 

The second option is based on what the database literature knows as
most-frequent-values histograms. This is a technique used by
PostgreSQL. The (over-)simplified version of this is to say in a VoID
description: 

\begin{verbatim}
_:foo void:propertyPartition 
  [
    void:property dct:language ;
    void:mostFrequentObjects lang:nb .
].
\end{verbatim}
This would declare \rdfterm{lang:nb} as the most frequent object for
\rdfterm{dct:language}, and this simple example would solve the
\langcase. 

While rare objects might more seldomly occur in queries, they may
contribute very significantly to the performance if they are joined
first if they do. Thus, it might be worthwhile to have a property
\rdfterm{void:leastFrequentOjects} too, as well as corresponding
properties for subjects.


However, this isn't a histogram, it merely mentioning the objects that
would most likely be useful. To get a histogram we would need a
similar \rdfterm{void:frequency} attribute and both this and
\rdfterm{void:mostFrequentObjects} would have to accept ordered
lists. 

Another option would be to do something like:
\begin{verbatim}
_:foo void:propertyPartition 
  [
    void:property dct:language ;
    void:mostFrequentObjects 
    [
      void:object lang:nb ;
      void:frequency "0.012" .
    ]
].
\end{verbatim}
but this solution would be less general but of similar complexity than
the nested solution above.

I find the prospect of writing a first paper to explore these different
extensions to VoID very appealing. I would need to put down some
substantial work on notation, but not intending an very elaborate
theory, which is something that requires \SRL. The paper could discuss
the different approaches above, their merits and drawbacks. After
publishing, I could work with the W3C Semantic Web Interest Group to
get one of the solutions into VoID.



\subsection{Fully indexed stores}

Some time ago, I wrote a partial paper about how fully indexed stores
like Hexastore \cite{hexastore} could easily find the cardinality of
any given triple pattern. It would be interesting to publish that soon
in some form.
