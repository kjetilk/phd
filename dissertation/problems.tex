\section{Problems}\label{sec:problems}

\subsection{Performance and stability problems}\label{sec:perfproblems}

\begin{problem}
The complexity of SPARQL is high.
\end{problem}

\subsection{Problems of shared understanding}\label{sec:semproblems}

\subsection{Problems with evaluations}\label{sec:evalproblems}

\subsection{Development problems}\label{sec:devproblems}

There are several issues that make Semantic Web development difficult,
that are connected with development in current programming languages
and current Semantic Web libraries.

This is very important, because even though we have argued that the
programmer needs to be eliminated from the data integration task,
there are several functions where they need to be involved, including
end-user application development, and in the short term, certainly
also the data integration task. 

Following the discussion in Section~\ref{sec:history}, the
availability of tools and examples so that the vast majority of active
Web developers can quickly publish and make use of RDF is a key to the
success of the Semantic Web.

As pointed out in \cite{darobin1}, there are few things that are
simpler in contemporary programming than to parse a string containing
a tree-formatted data structure into a tree that can be accessed
directly. However, RDF makes the assumption that the natural form of
data is not a tree, but a graph, and in the general case, it must be
addressed as such.

\begin{problem}\label{prob:graph}
Addressing graph data in a programming language.
\end{problem}

The initial learning by example by ``View Source'' of HTML markup can
be generalised to RDF by considering the term \emph{hypermedia}. The
most important aspect of hypermedia is, as argued in
\cite{Fielding_2000_Architectural-Styles}, that hypermedia can be used
to drive the interaction of applications, and that all the information
needed to do so is in the messages that are passed between the server
and the client, no information beyond that is needed. As such, the
importance of ``View Source'' extends well beyond the original
motivation of learning by example. More concretely, hypermedia can be
used to describe simple query interfaces, define how read-write
operations are to be made, etc. %TODO more here

\begin{problem}\label{prob:lapis}
Developers who are not well versed in RDF need a readily
understandable format that details how they can interact with an RDF
server that offers a read-write interface.
\end{problem}

\begin{problem}\label{prob:tpf}
Since the SPARQL language is described in an external specification,
it cannot be hypermedia.
\end{problem}


