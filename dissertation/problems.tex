\section{Problems}\label{sec:problems}

\subsection{Performance and stability problems}\label{sec:perfproblems}

\begin{problem}
The complexity of SPARQL is high.
\end{problem}

\subsection{Problems of shared understanding}\label{sec:semproblems}

\subsection{Problems with evaluations}\label{sec:evalproblems}

\begin{problem}
Evaluation methodologies are not founded on sound statistics.
\end{problem}

\begin{problem}
Few laws of the Semantic Web information space can reasonably be
expected to have validity far beyond the present.
\end{problem}


\begin{problem}
Evaluation methodologies has no foundation in philosophy of science.
\end{problem}


\subsection{Development problems}\label{sec:devproblems}

There are several issues that make Semantic Web development difficult,
that are connected with development in current programming languages
and current Semantic Web libraries.

This is very important, because even though we have argued that the
programmer needs to be eliminated from the data integration task,
there are several functions where they need to be involved, including
end-user application development, and in the short term, certainly
also the data integration task. 

Following the discussion in Section~\ref{sec:history}, the
availability of tools and examples so that the vast majority of active
Web developers can quickly publish and make use of RDF is a key to the
success of the Semantic Web.

As pointed out in \cite{darobin1}, there are few things that are
simpler in contemporary programming than to parse a string containing
a tree-formatted data structure into a tree that can be accessed
directly. However, RDF makes the assumption that the natural form of
data is not a tree, but a graph, and in the general case, it must be
addressed as such.

\begin{problem}\label{prob:graph}
Addressing graph data in a programming language.
\end{problem}

The initial learning by example by ``View Source'' of HTML markup can
be generalised to RDF by considering the term \emph{hypermedia}. The
most important aspect of hypermedia is, as argued in
\cite{Fielding_2000_Architectural-Styles}, that hypermedia can be used
to drive the interaction of applications, and that all the information
needed to do so is in the messages that are passed between the server
and the client, no information beyond that is needed. As such, the
importance of ``View Source'' extends well beyond the original
motivation of learning by example. More concretely, hypermedia can be
used to describe simple query interfaces, define how read-write
operations are to be made, etc. %TODO more here

\begin{problem}\label{prob:lapis}
Developers who are not well versed in RDF need a readily
understandable format that details how they can interact with an RDF
server that offers a read-write interface.
\end{problem}

\begin{problem}\label{prob:tpf}
Since the SPARQL language is described in an external specification,
it cannot be hypermedia.
\end{problem}

Until recently, a query engine would need to break down a SPARQL query
to individual triple pattern matches, which would then use for example
the \jcode{listStatements()} method of Jena, \jcode{filter()} method
of Sesame or \pcode{get\_statements} of \pmodule{RDF::Trine}, so that only
individual triple patterns could be evaluated against the underlying
triplestore.  The latter also had a \pcode{get\_pattern} method that
could be implemented if the underlying store had a way to optimise
Basic Graph Patterns. Sesame, on the other hand, allow implementations
to get a query representation, which it must then evaluate.
Nevertheless, this resulted in many problems:

\begin{problem}
By breaking down the query down to individual triple patterns, the
query engine cannot take advantage of optimisations that involve
multiple triple patterns or other parts of the query.
\end{problem}

\begin{problem}
If methods such as \pcode{get\_pattern} are added for every part of the
query, it would increase the complexity of the API dramatically.
\end{problem}

\begin{problem}
By merely passing the entire query representation to an
implementation, the burden of evaluating the entire query is also
transferred, which makes it harder to use default implementations for
most of the query.
\end{problem}

The same kind of problems occur in many operations, such as parsing
and serialisation, where an underlying implementation might have
relevant information to enhance e.g. the performance of an operation,
but where the API does not allow it to expose it so that upper layers
can take advantage of it.

Another type of problems occur when working with dynamic RDF that may
mix terminological and assertional information with object-oriented
programming. LITEQ \cite{leinberger2014semantic} has a modern approach
to a static case, where the developer gets substantial support from
their Integrated Development Environment when programming, and where
static typing is seen as a virtue that helps quality assurance of the
code.

However, a more interesting case is where an application (in a broad
sense) can adapt to the data it is seeing. For example, say that there
is an implementation of a Boat and a Car. It may not have been
anticipated when the application was first developed, but the
application sees an AmphibiousVehicle. Since it already knows how to
propel a Car and Boat, it should be able to adapt to the situation
given the context of whether the vehicle is on the road or in water.

\begin{problem}
Current programming paradigm deals poorly with dynamic RDF data
containing a mix of terminological and assertional information.
\end{problem}
