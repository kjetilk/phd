\section{Motivation}\label{sec:motivation}

The following section provides an overview of the motivation behind
this work in general and each direction in particular. Detailed
problem statements are given in Section~\ref{sec:problems}.

\subsection{Cruise altitude view}

There is a class of diverse, yet structurally similar, problems people
face nearly every day: When we book a flight, we may be interested in
a romantic getaway, yet we are forced to select the airport of
destination first. We often end up buying the same groceries because
we don't know what will taste well together. Last time I bought a car,
the two most important features was that it had 4 wheel drive and that
the trunk was long enough to fit a full-size baby stroller
length-wise. To find a property to build a home, I took into account
the sun conditions, that it was close to public transport that would
transport me to campus within a certain time frame, close to
ski tracks, low concentration of uranium in the ground, number of children for
my children to play with, and an affordable price level. And once the
house was being built, I had to choose the toilet seat depending on
the frame in the wall.

Of fundamental importance is that even though each of these
problems are rarely encountered, there are so many such problems that
we encounter them frequently. I would like to see a Semantic Web
where the plumber delivers his offer in a manner so that the open
options can be distinguished from the constraints, where a query can
be formulated to find the area I'd like to build my house based on
such extensive criteria, where the very rare comparison between a
collapsed baby stroller and car trunk can be done, where I will be
able to cook better food without being a chef and where I can book my
travel based on what I want to experience rather than make me choose a
destination based on lacking information.

Now, all this information exists on the Web, or at the very least, in
some database, and for the most part, the above problems can be
solved by surfing many websites. We are, however, reminded of ``Connolly's
Lament'' (due to Dan Connolly, one of the Semantic Web's early champions):
\begin{quote}
 The bane of my existence is doing things I know the computer could do
 for me.
\end{quote}

Moreover, the problem could be solved by writing programs to integrate
the variety of data sources for each scenario, but this is clearly not
sustainable, as each particular integration is of interest to very
few, but the cost of the integration increases for every integration
task. Fewer people to pay for a more expensive task quickly becomes
economically infeasible.

Technically, it might be an easier problem to solve if the data could
be centralized, and that solution has been argued for by
\cite{DBLP:conf/semweb/BetzGHS12}. Nevertheless, even if all their
arguments remain valid from a technical viewpoint, the most important
arguments against centralisation are social: If important components
of the Semantic Web are under the control of some central authority, it
is likely that one cannot publish anything, link to anything, or
develop any application without permission. It may create an unhealthy
power structure, which may create commercial or intellectual
hegemonies. Tim Berners-Lee discussed this in \cite{berners2000weaving} and an
interview in Wired Magazine in 2014\footnote{see
  \url{http://www.wired.co.uk/news/archive/2014-02/06/tim-berners-lee-reclaim-the-web}}.
This is the key motivation to still design a distributed Semantic Web,
regardless of the technical difficulties that may arise.

The Crosscloud\footnote{See \url{http://crosscloud.org/}, quote
  retrieved on 2016-05-30.} project is motivated from the same
observation:
\begin{quote}
Today, in a world of cloud-hosted software, every application is a
kind of trap. Even if it lets you export your data in a form other
systems can read, many of the best apps have social features. You
can't switch because your friends or colleagues are still using the
old site — the new site will be a ghost town. This stifles competion,
blocks innovation, and leaves users less happy with the systems they
are using. While some developers might want to lock-in users, we trust
that many value user happiness and would open their systems if it was
technologically practical. Our goal is to make it practical.
\end{quote}

The naive enthusiast may now forward the claim that Linked Data is the
solution and the whole solution, and at times of hubris, I may myself
have been guilty of conjecturing that sitting down to write the code
based on what we already know would solve the problem.

At times of doubt, the large number of unsolved problems, theoretical
as well as practical become apparent. In the following, I will try to
enumarate the problems I have foreseen, discuss their place in current
research, and what problems I have attempted to tackle in this
dissertation.

Both the idea of a decentralised Web and the idea that SPARQL
endpoints can be made openly available for any query on the Internet
seems to go against the conventional wisdom of the database
community. We are, however, motivated to solve this problem by the
social requirement of decentralisation and by the use cases mentioned
at the start of this section.

It is also important to frame the question correctly: We want to
answer queries, we do not require that any remote server accepts the
entire burden of evaluating queries. It is becoming quite clear that
the current practice of evaluating any arbitrary SPARQL query on a
remote server is economically unsustainable, and it is made clear by
the instability of SPARQL endpoints on the Web \cite{buil2013sparql}.

\subsection{Hypermedia}

The hypermedia idea is attractive as it is the extension of the ``View
Source'' idea that brought me to the Web. In its original form, I
believe that this feature is essential for broad developer
adoption\footnote{See also
  \url{http://archive.oreilly.com/pub/a/network/2000/04/13/CFPkeynote.html}
  for a view aligned with this.}. With hypermedia, a developer would
not need to read extensive documentation, but can start using the data
to create applications that interact with the data with a very low
barrier to entry. When designing the hypermedia discussed in
Section~\ref{sec:conlapis}, this was my primary design motivation. 

It is also possible to think of hypermedia in terms of machine
interaction. A classical SQL database is often the antithesis of
hypermedia, as the semantics of columns is usually not clear, to
interpret the schema, domain expertise is needed, the documentation
may reside in documents not even accessible to machines, for various
reasons.

RDF is well suited to include information that clarifies the semantics
with the data so it could be argued that most RDF documents would be
hypermedia, but what is often overlooked is that the documents need to
be explicit in linking schema from all documents. Moreover, control
information to enable anticipated and serendipitous applications must
be made available in the RDF. E.g. It is not sufficient for a pizza
restaurant to publish data on pizzas and link to the schema, the RDF
must also express how to order to enable a software agent to organise
a party.

\subsection{Developer friendliness}

As noted, my hypermedia motivation came from the perspective as a
developer, but hypermedia has a substantial research literature behind
it. While software engineering certainly is a large field, the
direction of work concerning developer friendliness comes mainly from
personal experience, from developing solutions in the industry, but
also from working with other developers who are at best indifferent to
Semantic Web technologies, and in many cases antagonistic towards it. 
As noted in Section~\ref{sec:history}, Aaron Swartz echoed the
sentiment already a decade ago that the Semantic Web had failed
because it did not provide tangible benefits, and had gotten a
reputation for being unfriendly to developers. It is crucial to
address this problem.


\subsection{Philosophy of Science and statistical methods}

While a failure to be developer friendly may be one reason why the
Semantic Web has not succeeded at a grand scale, my training in
natural science lead me to ask if there may be other reasons that we
do not see the remarkable rate of progress of the natural sciences in
this field. These questions are as hard to ask and to live with as to
answer. 

I became early quite confident that the practice of benchmarking was
an easy target, with its lack of summary statistics and inability to
support proper hypothesis testing, standardising to deal with
complexity rather than methodological advances, and lack of ability to
formulate the experiment as severe enough.

\subsection{Caching in the Internet}\label{sec:motivcache}

Clearly, the above motivation points out federated SPARQL as a key
solution to the problem of query answering over decentralised sources,
and that was originally where I intended my main contributions. During
the work, it became clear that the stability of individual endpoints
would need to addressed first. In the detailed problem descriptions,
this will be further discussed as the choice of caching as a point of
focus is somewhat arbitrary.

An important motivation was not problems, however, but opportunity:
The proliferation of Content Delivery Networks is an opportunity for
improvement, the emergence of Triple Pattern Fragments, and traits
based planning in Section~\ref{sec:conpush} all represent important
opportunities to seize.


It is crucial, however, that we note that the Internet infrastructure
is accommodating for solving parts of this problem with the presence
of caches at several different levels. We would need to create an
infrastructure to take advantage of these caches, and encourage data
publishers to include metadata to allow caches to be managed in a
standard-defined way. The scale at which this can already be done was
at first important to understand.

We further note that some clients also have significant processing
capacity and could process SPARQL queries themselves. Others, like
microcontrollers on the Internet of Things, cannot. This calls for a
diverse infrastructure that can accommodate all kinds of
clients. Nevertheless, these are among the factors that
could make SPARQL query answering on the open Web practically
feasible. 

The convergence of all the directions of this work was intended to be
a contribution to ease the burden of servers by allowing other parties
to take parts of the query evaluation.



%%  LocalWords:  hegemonies
