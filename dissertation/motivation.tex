\section{Motivation}\label{sec:motivation}

There is a class of diverse, yet structurally similar, problems people
face nearly every day: When we book a flight, we may be interested in
a romantic getaway, yet we are forced to select the airport of
destination first. We often end up buying the same groceries because
we don't know what will taste well together. Last time I bought a car,
the two most important features was that it had 4 wheel drive and that
the trunk was long enough to fit a full-size baby stroller
length-wise. To find a property to build a home, I took into account
the sun conditions, that it was close to public transport that would
transport me to campus within a certain time frame, close to
ski tracks, abundance of uranium in the ground, number of children for
my children to play with, and an affordable price level. And once the
house was being built, I had to choose the toilet seat depending on
the frame in the wall and even a toilet seat that would fit.

Of fundamental importance is that is that even though each of these
problems are rarely encountered, there are so many such problems that
we encounter them frequently. I would like to see a Semantic Web
where the plumber delivers his offer in a manner so that the open
options can be distinguished from the constraints, where a query can
be formulated to find the area I'd like to build my house based on
such extensive criteria, where the very rare comparison between a
collapsed baby stroller and car trunk can be done, where I will be
able to cook better food without being a chef and where I can book my
travel based on what I want to experience rather than make me choose a
destination based on lacking information.

Now, all this information exists on the Web, or at the very least, in
some database, and for the most part, the above problems can be
solved by surfing many websites. We are, however, reminded of ``Connolly's
Lament'' (due to Dan Connolly, one of the Semantic Web's early champions):
\begin{quote}
 The bane of my existence is doing things I know the computer could do
 for me.
\end{quote}

Moreover, the problem could be solved by writing programs to integrate
the variety of data sources for each scenario, but this is clearly not
sustainable, as each particular integration is of interest to very
few, but the cost of the integration increases for every integration
task. Fewer people to pay for a more expensive task quickly becomes
economically infeasible.

Technically, it might be an easier problem to solve if the data could
be centralized, and that solution has been argued by
\cite{DBLP:conf/semweb/BetzGHS12}. Nevertheless, even if all their
arguments remain valid from a technical viewpoint, the most important
arguments against centralisation are social: If important components
of the Semantic Web is under the control of some central authority, it
is likely that one cannot publish anything, link to anything, or
develop any application without permission. It may create an unhealthy
power structure, which may create commercial or intellectual
hegemonies. Tim Berners-Lee discussed this in \cite{TODO} and an
interview in Wired Magazine in 2014\footnote{see
  \href{http://www.wired.co.uk/news/archive/2014-02/06/tim-berners-lee-reclaim-the-web}}.
This is the key motivation to still design a distributed Semantic Web,
regardless of the technical difficulties that may arise.

The naive enthusiast may now forward the claim that Linked Data is the
solution and the whole solution, and at times of hubris, I may myself
have been guilty of conjecturing that sitting down to write the code
based on what we already know would solve the problem.

At times of doubt, the large number of unsolved problems, theoretical
as well as practical become apparent. In the following, I will try to
enumarate the problems I have foreseen, discuss their place in current
research, and what problems I have attempted to tackle in this
dissertation.

Both the idea of a decentralised Web and the idea that SPARQL
endpoints can be made openly available for any query on the Internet
seems to go against the conventional wisdom of the database
community. We are, however, motivated to solve this problem by the
social requirement of decentralisation and by the use cases mentioned
at the start of this section.

It is also important to frame the question correctly: We want to
answer queries, we do not require that any remote server accepts the
entire burden of evaluating queries. It is becoming quite clear that
the current practice of evaluating any arbitrary SPARQL query on a
remote server is economically unsustainable, and it is made clear by
the instability of SPARQL endpoints on the Web \cite{TODO}.

However, we note that the Internet infrastructure is accommodating for
solving parts of this problem with the presence of caches at several
different levels. Some clients also have significant processing
capacity, and these are among the factors that could make SPARQL query
answering on the open Web practically feasible. The focus of this work
is to ease the burden of servers by allowing other parties to take
part of the query evaluation.



%%  LocalWords:  hegemonies
