\section{Overview}\label{sec:introoverview}

The World Wide Web, or just the Web for short, is a well-known global
information space invented by Tim Berners-Lee in 1989, further emerged
in the 1990-ties. It is characterised first and foremost by its
universality, anyone can set up a computer, connect it to the Internet and
start serving data or documents from it, and further adapt it to their purpose. 

The Semantic Web is an extension of the Web that has been under
development since 1997\footnote{The first working group draft of the
  RDF specification is dated 1997-08-01, see
  \url{http://www.w3.org/TR/WD-rdf-syntax-971002/}.}, to extend the
Web with languages for expressing information in a machine processable
form\cite{semwebroadmap}.



The present work sits in the confluence of several contemporary
efforts in the Semantic Web community. The focus is on query answering
with the SPARQL query language, with emphasis on exploiting the World
Wide Web, but it touches upon query federation, hypermedia, empiricial
methods for evaluating performance, standards compliance and even
philosophy of science. With a scope of this breadth, the investigation
is shallow, it is hoped that some key insights will provoke further
investigation rather than being the end of any conversations.

This chapter serves to tie the published papers into a whole, provide
an overarching rationale and problem statements, overview of the
contributions and a discussion of the problematic sides of the study
seen from a philosophy of science viewpoint. It is structured as
follows: Section~\ref{sec:nodef} discusses the possibility of defining
the Semantic Web, Section~\ref{sec:objectivity} discusses the problem
of maintaining objectivity, Section~\ref{sec:history} attempts to
satisfy the previous discussion by detailing the author's personal
involvement in the Semantic Web. Section~\ref{sec:motivation} has an
overview of the motivation behind this work.
Section~\ref{sec:related} puts the present work into the context
within the existing literature. Section~\ref{sec:problems} summarises
important research problems and details the motivation, and
Section~\ref{sec:papersincontext} details how the contributions
address these problems and discusses weaknesses of the papers with the
benefit of hindsight. Key insights are finally summarised in
Section~\ref{sec:conclusions}.\todo{Big problem statement here}


\section{Semantic Web -- A Definition?}\label{sec:nodef}

The Semantic Web can be described as a machine-readable Web of Data,
defined by a technology stack managed by the World Wide Web
Consortium. While individual technologies can be given precise,
operational definitions, the above definition is naive and can hardly
be used operationally to derive new knowledge. The Semantic Web is a
complex, human artefact that must be understood not only in technical
terms, but also in social, economic and cultural terms. First, the Web
is not only readable, but also writeable. Whether the essential parts
of a data integration problem will be solved by e.g. pervasive
ontology alignment, a linked open vocabularies approach where many
players adopt a fairly consistent set of vocabularies, or by major
players that are able to force convergence towards e.g. \texttt{Schema.org}
is an example of the social, economic and cultural mechanisms that
shape the future of the Semantic Web, see
Section~\ref{sec:semproblems} for a more elaborate
discussion. Likewise, it may not be the W3C Semantic Web stack at all
that achieves success, though unlikely at this point, it may be
something in the extension of microformats\footnote{See
  \url{http://microformats.org/}}, microdata\cite{microdata}, or
something else entirely. For a thoughtful criticism of Linked Data
with a proposed alternative, see \cite{darobin1}, but again, it is
likely that social, economic and cultural factors will be as decisive
as technical. One may say that e.g. microformats cannot help achieve
the visions that may be stated for the Semantic Web, but that again
emphasises visions, not operational definitions.

One such vision is the article by Berners-Lee, Hendler and Lassila in
Scientific American 2001\cite{berners2001semantic}, but it does not attempt such an
operational definition. Neither \cite{semwebroadmap},
\cite{berners2000weaving} nor \cite{Allemang:2008:SWW:1386668} contain
such definitions. However, it is clear that an essential
characteristic of the Web, and by extension, the Semantic Web, is its
\emph{universality} and therefore, any actor should be as free as
possible to adapt it to their visions and needs. A pursuit of a
definition of the term ``Semantic Web'' would require a philosophical
treatment which would be beyond the scope of this thesis, and even
with that, may not be a fruitful exercise from a technical
perspective. 

Therefore, rather than attempt a definition, understanding is better
promoted by following the example of \cite{berners2001semantic} and
declaring a vision.


\section{Objectivity of the Study}\label{sec:objectivity}

A vision has necessarily a strong personal element. This challenges
the ideals of objectivity in scientific investigations. A longer
discussion of the desirability and attainability of objectivity in
science is beyond the scope of this thesis, but objectivity can be
challenged from many different angles
\cite{sep-scientific-objectivity}.

The problems are also discussed throughout
\cite{chalmers1999whatis}. It is interesting to consider the example
on page 25, where the author describes a method forwarded by Galileo
Galilei to measure the diameter of a star. The method was objective in
the sense if followed, it would yield the same value today as it did
when Galileo employed it. However, it relied on faulty assumptions and
is therefore invalid. The same kind of problem may occur in the
evaluation of e.g. SPARQL engines. If a benchmark relies on a faulty
workload, the application of the benchmark may seem objective to the
investigator, but its external validity should obviously be
challenged. Moreover, if the benchmark relied on a use case supplied
by e.g. an industrial partner, the validity could be further
challenged based on e.g. the restriction on viewpoint that this
represents. 

The impact of the Web in particular and information technologies in
general on society is great, and research projects are often funded
because of their relevance to society. With this perspective, this
research is hardly value-free. As individual researchers, we may
develop a strong relationship to the systems we create, as we have a
great intellectual investment in them. This may influence our ability
to reject a flawed hypothesis even if we should. Even though we note
in the discussion of the contributions in Section~\ref{sec:conphil}
the weaknesses of Thomas Kuhn's philosophy of science, his insistence
of understanding science as a social endeavour stands firm.

These challenges are inescapable, and therefore, it is important that
the researcher acknowledges such problems and explicitly details their
own personal motivations, visions and possible biases to let the
reader decide whether the researcher's subjective beliefs or other
external pressures has influenced the validity of their
contribution. It is also important to be humble and admit that even
though we as researchers treasure our objectivity, like Galileo, we
may not be in the ideal position to judge our own objectivity, as
should be clear from the few challenges that research methods are
getting in the literature. We should therefore detail our background
to better equip the reader to judge.

I shall attempt to do so in the following, but also note that this
influences the writing style. There are several conventions of
scientific writing that stands in opposition to the goal of letting
the reader understand the possible influence of subjective
beliefs. The frequent use of passive voice is one such convention,
that in my opinion tries to hide subjectivity rather than alleviate
its harmful effects. If the methodology used in the investigation is
flawed, it should not be obscured by writing style, the methodology
will need to be enhanced. Double blind trials in medicine are an
example of why it is important to acknowledge the investigator's
role. That is not to say that the use of passive voice is always
inappropriate, when the role of the investigator cannot be
misunderstood, it can be used just like any other language construct.

Of less importance is the use of ``we'', when it is clear that the
work has been done by a single investigator. I hold as a general
principle that the role of the investigator should never be obscured,
and therefore, such use is inappropriate. The use of pronouns should
not differ substantially in scientific writing from other writing, and
I will use them accordingly. In particular, I will use ``we'' when
more than one person has been involved, or when the reader is included
(e.g. ``We note that...'').

Lastly, while the discussion in Section~\ref{sec:conphil} should make
clear that I think that there are epistemological problems underneath
the entire field of study, i.e. there is reason to question the
validity of any result, there are invariably certain statements that
have a particularly weak foundation, especially in the motivation. I
shall be careful to use the term ``belief'' to qualify such
statements.

\section{A Personal History}\label{sec:history}

As noted in the previous section, objectivity may be challenged by a
researcher's intellectual investment in a certain field of study. I
admit that I have a very large intellectual investment in the success
of the Semantic Web. This section serves two purposes: It should give
the reader a good understanding of this intellectual investment, and
enable them to assess the impact of any subjective beliefs. Secondly,
it serves to clarify \emph{my} vision, which we noted in
Section~\ref{sec:nodef} is more important than to attempt a
definition. The vision strongly influences the motivation detailed in
Section~\ref{sec:motivation}.


I created my first Web page in December 1994, and strongly appreciated
the value of the fact that I could easily contribute to the Web, and
therefore, I quickly posted anything that I had of value, for example
a collection of my best mountaineering pictures, constrained not by
the platform, but by my own time. Initially, I learnt purely by
example, i.e. using the ``View Source'' menu item when I encountered
web pages that I liked. I also had the power to mint identifiers. 

With identifiers, and the hypertext language HTML came the fact that I
could link to anyone, and anyone could link to me was highly
empowering: Discovering that people I didn't know linked to my
material was a strong, social reward. After some time, I had tens of
thousands of visitors every month.

In 1996, I started the Web pages of the \textit{Norwegian Skeptics Society}
\texttt{skepsis.no}, and therefore got the control over an entire Web
server. That was another revelation, as I could actually run code, and
from that arose the need to not just learn by ``View Source'', but
start to read the specifications, notably HTML, CSS and
HTTP. Appreciating the design of these specifications took a while,
but eventually, the value of orthogonal specifications and separation
of concerns became apparent. Also, at the same time, I realised that I
could be standing on the shoulders of giants by using Free Software.

When the RDF working drafts were first published, I was first
overwhelmed by the amount of new text to read, but at the same time I
had a growing realisation that getting information out was not solving
the problem \textit{The Norwegian Skeptics Society} needed to solve: We needed
to push information into closed minds. My initial plan to address that
was to enable a higher degree of targeting of information, and a Web-wide
conversation. To do that, I wanted to create a large thesaurus of
topics of interest, and create annotations of the type ``this article
is a rebuttal of that article'', and an index that browsers could
query, so that when a user viewed an article, they would also get a
critical context. At the same time, Netscape was open-sourced as
Mozilla, and so, I was quite convinced I could contribute the code
needed for this to be accepted. 

I shared these ideas on some mailing lists in August 1998, and one of
the persons who responded was Dan Brickley, who would chair the RDF
working group. He quickly convinced me that RDF was not something to
be afraid of, quite the contrary, it was exactly what I needed.

However, all the above had happened on my spare time, as the topic of
my study was something else entirely: Cosmology. Therefore, the
interest lay dormant for a few years as I finished my
Cand. Scient.-degree.  After that, I got some limited funding to write
what today would be called a Semantic Content Management System. The
project was far too ambitious and largely a failure, but it got me
some valuable experience with Semantic Web technology.

I was convinced that the things that made the Web great to work with
had to be present in the Semantic Web as well: Everybody is empowered
to publish, find and quickly make use of Free Software, learn by
``View Source'' (an idea which becomes even more powerful with
semantics, as if you can understand what you are viewing simply by
looking at the message, you can begin processing it), anyone can mint
identifiers, anybody can reuse those identifiers in their own
contexts, and eventually graduate to reading technical materials as
needed. Many of these things are virtues of the Semantic Web as it is
an extension of the Web, others have not been sufficiently cultivated
or even appreciated. Moreover, my experience corresponds to the
virtues that Tim Berners-Lee has attributes the Web's success
to\footnote{see
  \url{http://dig.csail.mit.edu/2007/03/01-ushouse-future-of-the-web}}.

It was not until I joined Opera Software in 2005 to work on the now
defunct Opera Community social networking site, that I had a bit more
time to work on Semantic Web technology. When I joined, they had
rudimentary support for an RDF/XML-based ``Friend of a Friend'' (FOAF)
export, which I improved before the initial release. At the time, it
was not anticipated that everything would have a URI, so Dan Brickley
was a proponent of the view that most things, including people, would
be identified mainly by their properties, e.g. their email address. It
was a certain tension between that view and the ``give everything a
URI''-movement heralded by Tim Berners-Lee at the time, but I decided
to place myself in between, by giving everything a URI, but at the
same time ensure that enough properties would identify people. I also
included outward links, so that people could integrate the database
export we provided with their hand-maintained FOAF. I also added
support for photo gallery metadata and linking tags as given in a
personal SKOS-ontology to Wordnet to enable assigning clearer meaning
to tags. This effort was acknowledged in the Linked Data Design
Issue\cite{linkeddataissue} that formed the basis for the more recent Linking
Open Data project.

However, at this point, I started to realise that it was difficult to
develop applications around traversing Linked Data, and that
prompted me to consider the possibility to offer the ability to
execute arbitrary queries. On 2005-11-30, I published the first public
SPARQL endpoint with actual production data\footnote{see
  \url{http://www.onlamp.com/pub/wlg/8609}}, consisting of 2695114
triples, but it quickly grew to approx. 15~million triples.

Unfortunately, management did not permit more time to enhance and document the
SPARQL Endpoint further, and it didn't see much practical use. 
I was, however, allowed to join several World Wide Web Consortium
groups. First, the Web Content Label Incubator Group, which was tasked
with discussing a replacement for the then archaic Platform for
Internet Content Selection, and resolved to use RDF for this
purpose, and it transitioned into the POWDER Working Group, which
provided the specifications. 

More importantly, I joined the Semantic Web Education and Outreach
(SWEO) Interest Group, where I admitted my frustration with the lack
of practical progress with Semantic Web technologies and the relative
lack of uptake. This sentiment was echoed by the late Aaron Swartz,
who was asked by the group on his opinion, and he emailed the
following statement\footnote{see
  \url{https://lists.w3.org/Archives/Public/public-sweo-ig/2006Dec/0138.html}}:
\begin{quote}
I'm not sure what SWEO is, but my feeling is and pretty much always
has been that the Semantic Web people need to start putting together
Genuinely Useful stuff that can be done Right Now.
\end{quote}

This was one of the motivations I had for starting the SWEO Community
Projects, where a questionnaire was posted on the Web, and people from
the Semantic Web community were challenged to come up with a project
that would provide practical benefits in the short term. We received
10 proposals, out of which 3 was selected for backing by the SWEO
IG. None of them were successful, but a fourth proposal, submitted by
Chris Bizer and Richard Cyganiak built momentum quickly and was
therefore also selected for backing, despite some criticism by several
IG members, myself included. The proposal was titled ``Linking Open
Data'' and sought to take already abundant open data and model it
using guidelines of the Linked Data Design
Issue\cite{linkeddataissue}. LODstats\cite{auer2012lodstats} provides
extensive statistics on the results of this project, and has of this
writing seen nearly 10~000 data sets.

At this point, I left Opera to work as a consultant on Semantic Web
technologies. One of the projects that were successfully completed was
called Sublima\cite{sublima}. It was driven entirely by RDF and SPARQL,
and featured faceted navigation, navigation in a thesaurus, and full
text search. Experiences from this project influenced requirements of
SPARQL 1.1 \cite{sparql11query}, and I was an editor of the SPARQL 1.1 Features and
Rationale specification\cite{sparql11new}. Parts of the property paths
feature and aggregate queries were used extensively. Write operations
were also implemented with SPARQL using what was then specified only
in a member submission to the W3C\cite{seaborne2008sparql}. This was the basis for
the update language in SPARQL 1.1 but differed substantially in
surface syntax, partly because our work showed how certain usage patterns
could be simplified. 

Two features that arose from Sublima requirements were not accepted
for SPARQL 1.1, however: Full text index and a feature known as
``Limit Per Resource''\footnote{see
  \url{http://www.w3.org/2009/sparql/wiki/Feature:LimitPerResource}}. The
latter feature arises from the fact that usually, the user is
indifferent to the number of solutions to a query, and therefore
limiting by the number of solutions is unhelpful. This is especially
true since other features of SPARQL, such as optional clauses, are
helpful in dealing with heterogeneous data. In the Sublima case, it
was interesting to limit by the number of articles that were returned,
but the number of solutions depended on the number of authors of an
article and the concepts used to classify an article. To achieve the
desired effect with SPARQL 1.1, one would have to write a very complex
subquery, which is undesirable for such an important feature.


Another project, based on much of the same code
base replaced the SKOS ontology of the Sublima project with an OWL
ontology, and also used reasoning to aid navigation in multimedia
libraries.



However, it became clear that I could not have sufficient time in the
industry to pursue what was developing as my main interest, query
answering over data on the open Web for generic application
development. 
