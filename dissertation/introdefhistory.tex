The present work sits in the confluence of several contemporary
efforts in the Semantic Web community. The focus is on query answering
with the SPARQL query language, with emphasis on exploiting the World
Wide Web, but it touches upon query federation, hypermedia, empiricial
methods for evaluating performance, standards compliance and even
philosophy of science. With a scope of this width, the investigation
is shallow, it is hoped that some key insights will provoke further
investigation rather than being the end of any conversations.

\section{Semantic Web -- A definition?}

The Semantic Web can be described as a machine-readable Web of Data,
defined by a technology stack defined by the World Wide Web
Consortium. While individual technologies can be given precise,
operational definitions, the above definition is naive and can hardly
be used operationally to derive new knowledge. The Semantic Web is a
complex, human artefact that must be understood not only in technical
terms, but also in social, economic and cultural terms. First, the Web
is not only readable, but also writeable. Whether the essential parts
of a data integration problem will be solved by e.g. pervasive
ontology alignment, a linked open vocabularies approach where many
players adopt a fairly consistent set of vocabularies, or by major
players are able to force convergence towards e.g. \texttt{Schema.org}
is an example of the social, economic and cultural mechanisms that
shape the future of the Semantic Web. Likewise, it may not be the W3C
Semantic Web stack at all that achieves success, it may be
microformats\footnote{see e.g. }, microdata\cite{TODO}, or something
else entirely\footnote{for a thoughtful criticism of Linked Data with
  a proposed alternative, see \href{http://berjon.com/linked-data/}},
but again, it is likely that social, economic and cultural factors
will be as decisive as technical. One may say that e.g. microformats
cannot help achieve the visions that may be stated for the Semantic
Web, but that again emphasises visions, not operational definitions.

One such vision is the article by Berners-Lee, Hendler and Lassila in
Scientific American 2001\cite{TODO}, but it does not attempt such an
operational definition. Neither \cite{TODO} nor \cite{TODO} contains
such definitions. However, it is clear that an essential
characteristic of the Web, and by extension, the Semantic Web is its
\emph{universality} and therefore, any actor should be as free as
possible to adapt it to their visions and needs, and so, a pursuit of
such a definition may not be desirable. Therefore, rather than
defining it, one should carefully declare one's visions.

\section{A personal history}


