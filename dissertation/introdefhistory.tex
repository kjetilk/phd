\section{Overview}

The World Wide Web, or just the Web for short, is a well-known global
information space invented by Tim Berners-Lee in 1989, further emerged
in the 1990-ties. It is characterised first and foremost by its
openness, anyone can set up a computer, connect it to the Internet and
start serving data or documents from it. 

The Semantic Web is an extension of the Web that has been under
development since 1997\footnote{The first working group draft of the
  RDF specification is dated 1997-08-01, see
  \href{http://www.w3.org/TR/WD-rdf-syntax-971002/}.}, to extend the
Web with languages for expressing information in a machine processable
form\cite{TODO}.



The present work sits in the confluence of several contemporary
efforts in the Semantic Web community. The focus is on query answering
with the SPARQL query language, with emphasis on exploiting the World
Wide Web, but it touches upon query federation, hypermedia, empiricial
methods for evaluating performance, standards compliance and even
philosophy of science. With a scope of this width, the investigation
is shallow, it is hoped that some key insights will provoke further
investigation rather than being the end of any conversations.

\section{Semantic Web -- A definition?}

The Semantic Web can be described as a machine-readable Web of Data,
defined by a technology stack defined by the World Wide Web
Consortium. While individual technologies can be given precise,
operational definitions, the above definition is naive and can hardly
be used operationally to derive new knowledge. The Semantic Web is a
complex, human artefact that must be understood not only in technical
terms, but also in social, economic and cultural terms. First, the Web
is not only readable, but also writeable. Whether the essential parts
of a data integration problem will be solved by e.g. pervasive
ontology alignment, a linked open vocabularies approach where many
players adopt a fairly consistent set of vocabularies, or by major
players are able to force convergence towards e.g. \texttt{Schema.org}
is an example of the social, economic and cultural mechanisms that
shape the future of the Semantic Web. Likewise, it may not be the W3C
Semantic Web stack at all that achieves success, it may be
microformats\footnote{see \href{http://microformats.org/]},
microdata\cite{TODO}, or something else entirely\footnote{for a
  thoughtful criticism of Linked Data with a proposed alternative, see
  \href{http://berjon.com/linked-data/}}, but again, it is likely that
social, economic and cultural factors will be as decisive as
technical. One may say that e.g. microformats cannot help achieve the
visions that may be stated for the Semantic Web, but that again
emphasises visions, not operational definitions.

One such vision is the article by Berners-Lee, Hendler and Lassila in
Scientific American 2001\cite{TODO}, but it does not attempt such an
operational definition. Neither \cite{TODO} nor \cite{TODO} contains
such definitions. However, it is clear that an essential
characteristic of the Web, and by extension, the Semantic Web, is its
\emph{universality} and therefore, any actor should be as free as
possible to adapt it to their visions and needs, and so, a pursuit of
such a definition may not be desirable. Therefore, rather than
defining it, one should carefully declare one's visions.

\section{A personal history}

I created my first Web page in December 1994, and strongly appreciated
the value of the fact that I could easily contribute to the Web, and
therefore, I quickly posted anything that I had of value, for example
a collection of my best mountaineering pictures, constrained not by
the platform, but by my own time. Initially, I learnt purely by
example, i.e. using the ``View Source'' menu item when I encountered
web pages that I liked. I also had the power to mint identifiers. 

With identifiers, and the hypertext language HTML came the fact that I
could link to anyone, and anyone could link to me was highly
empowering: Discovering that people I didn't know linked to my
material was a strong, social reward. After some time, I had tens of
thousands of visitors every month.

In 1996, I started the Web pages of the Norwegian Skeptics Society
\texttt{skepsis.no}, and therefore got the control over an entire Web
server. That was another revelation, as I could actually run code, and
from that arose the need to not just learn by ``View Source'', but
start to read the specifications, notably HTML, CSS and
HTTP. Appreciating the design of these specifications took a while,
but eventually, the value of orthogonal specifications and separation
of concerns became apparent.

When the RDF working drafts were first published, I was first
inundated by the amount of new text to read, but at the same time I
had a growing realisation that getting information out was not solving
the problem The Norwegian Skeptics Society needed to solve: We needed
to push information into closed minds. My initial plan to address that
was to enable a higher degree of targeting of information, and a Web-wide
conversation. To do that, I wanted to create a large thesaurus of
topics of interest, and create annotations of the type ``this article
is a rebuttal of that article'', and an index that browsers could
query, so that when a user viewed an article, they would also get a
critical context. At the same time, Netscape was open-sourced as
Mozilla, and so, I was quite convinced I could contribute the code
needed for this to be accepted. 

I shared these ideas on some mailing lists in August 1998, and one of
the persons who responded was Dan Brickley, who would chair the RDF
working group. He quickly convinced me that RDF was not something to
be afraid of, quite the contrary, it was exactly what I needed.

However, all the above had happened on my spare time, as the topic of
my study was something else entirely: Cosmology, and so the interest
laid dormant for a few years as I finished my Cand. Scient.-degree. 
After that, I got some limited funding to write what today would be
called a Semantic Content Management System. The project was far too
ambitious and largely a failure, but it got me some valuable
experience with Semantic Web technology.

It was not until I joined Opera Software in 2005 to work on the now
defunct Opera Community social networking site, that I had a bit more
time to work on Semantic Web technology. When I joined, they had
rudimentary support for an RDF/XML-based ``Friend of a Friend'' (FOAF)
export, which I improved before the initial release. At the time, it
was not anticipated that everything would have a URI, so Dan Brickley
was a proponent of the view that most things, including people, would
be identified mainly by their properties, e.g. their email address. It
was a certain tension between that view and the ``give everything a
URI''-movement heralded by Tim Berners-Lee at the time, but I decided
to place myself in between, by giving everything a URI, but at the
same time ensure that enough properties would identify people. I also
included outward links, so that people could integrate the database
export we provided with their hand-maintained FOAF. I also added
support for photo gallery metadata and linking tags as given in a
personal SKOS-ontology to Wordnet to enable assigning clearer meaning
to tags. This effort was acknowledged in the Linked Data Design
Issue\cite{TODO} that formed the basis for the more recent Linking
Open Data project.

However, at this point, I started to realise that it was difficult to
develop applications around traversing the Linked Data, and that
prompted me to consider the possibility to offer the ability to
execute arbitrary queries. On 2005-11-30, I published the first public
SPARQL endpoint with actual production data\footnote{see
  \href{http://www.onlamp.com/pub/wlg/8609}}, consisting of 2695114
triples, but it quickly grew to \~15~million triples.

Unfortunately, it wasn't allotted time to enhance and document the
SPARQL Endpoint further, and it didn't see much practical use. 
I was, however, allowed to join several World Wide Web Consortium
groups. First, the Web Content Label Incubator Group, which was tasked
with discussing a replacement for the then archaic Platform for
Internet Content Selection, and resolved to use RDF for this
purpose, and it transitioned into the POWDER Working Group, which
provided the specifications. 

More importantly, I joined the Semantic Web Education and Outreach
(SWEO) Interest Group, where I admitted my frustration with the lack
of practical progress with Semantic Web technologies and the relative
lack of uptake. This sentiment was echoed by the late Aaron Swartz,
who was asked by the group on his opinion, and he emailed the
following statement\footnote{see
  \href{https://lists.w3.org/Archives/Public/public-sweo-ig/2006Dec/0138.html}}:
\begin{quote}
I'm not sure what SWEO is, but my feeling is and pretty much always
has been that the Semantic Web people need to start putting together
Genuinely Useful stuff that can be done Right Now.
\end{quote}

This was one of the motivations I had for starting the SWEO Community
Projects, where a questionnaire was posted on the Web, and people from
the Semantic Web community were challenged to come up with a project
that would provide practical benefits in the short term. We received
10 proposals, out of which 3 was selected for backing by the SWEO
IG. None of them was successful, but a fourth proposal, submitted by
Chris Bizer and Richard Cyganiak built momentum quickly and was
therefore also selected for backing, despite some criticism by several
IG members, myself included. The proposal was titled ``Linking Open
Data'' and sought to take already abundant open data and model it
using guidelines of the Linked Data Design
Issue\cite{TODO}. LODstats\cite{TODO} provides extensive statistics on
the results of this project, and has of this writing seen nearly
10~000 data sets.

At this point, I left Opera to work as a consultant on Semantic Web
technologies. One of the projects that were successfully completed was
called Sublima\cite{TODO}. It was driven entirely by RDF and SPARQL,
and featured faceted navigation, navigation in a thesaurus, and full
text search. Experiences from this project influenced requirements of
SPARQL 1.1, and I was an editor of the SPARQL 1.1 Features and
Requirements specification\cite{TODO}. Parts of the property paths
feature and aggregate queries were used extensively. Write operations
were also implemented with SPARQL using what was then specified only
in a member submission to the W3C\cite{TODO}. This was the basis for
the update language in SPARQL 1.1 but differed substantially in
surface syntax, partly because our work showed how certain patterns
could be simplified. 

Two features that arose from Sublima requirements were not accepted
for SPARQL 1.1, however: Full text index and a feature known as
``Limit Per Resource''\footnote{see
  \href{http://www.w3.org/2009/sparql/wiki/Feature:LimitPerResource}}. The
latter feature arises from the fact that usually, the user is
indifferent to the number of solutions to a query, and therefore
limiting by the number of solutions is unhelpful. This is especially
true since other features of SPARQL, such as optional clauses, are
helpful in dealing with heterogeneous data. In the Sublima case, it
was interesting to limit by the number of articles that were returned,
but the number of solutions depended on the number of authors of an
article and the concepts used to classify an article. To achieve the
desired effect with SPARQL 1.1, one would have to write a very complex
subquery, which is undesirable for such an important feature.


Another project, based on much of the same code
base replaced the SKOS ontology of the Sublima project with an OWL
ontology, and also used reasoning to aid navigation in multimedia
libraries.



However, it became clear that I could not have sufficient time in the
industry to pursue what was developing as my main interest, query
answering over data on the open Web, for generic application
development. 16 years of experience thus motivates the research done
in this thesis.

