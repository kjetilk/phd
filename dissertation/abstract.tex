
\begin{abstract}
  This dissertation discusses a broad range of problems concerning the
  use of the SPARQL query language on the open, public Web. It is
  motivated from seeing decentralisation of data and infrastructure as
  an important social goal, and SPARQL as an enabling technology to
  solve problems using the Semantic Web. The dissertation makes
  contributions in hypermedia, where RDF is used to create a format
  that can tell humans and machines alike how to manipulate resources
  on the Web; philosophy of science, where important foundational
  problems around how to create valid knowledge and objectivity are
  discussed; statistical methods that better satisfies the
  requirements from philosophy of science than current practice; how
  to improve developer efficiency with novel programming paradigms;
  and finally how caching infrastructure in the Internet may be used
  to make query answering across the Web more robust.

\end{abstract}
