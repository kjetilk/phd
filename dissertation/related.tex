\section{Related work}\label{sec:related}

There is a substantial amount of relevant literature and ongoing
research that is well aligned with the present work, either because
the authors share parts of the motivation, or more serendipitous
reuse.

Verborgh et al. \cite{TODO} share many of the observations and goals
that motivates this study. They introduce a hypermedia system that can
be used to answer individual triple patterns (known as Triple Pattern
Fragments (TPF)). Their stated goal is to transfer as much as possible
of the burden of evaluating a query from the server to the
client. This can happen because a simple pattern match for a single
triple pattern need not have a SPARQL parser, nor a query
planner. Moreover, the server may materialise responses to frequent
triple patterns and store them in a file system, which can be managed
by a simple Web server. The results may also be paged, so that each
individual response may be kept small. However, TPF mandates that
every response must contain a cardinality estimate for every triple
pattern, an operation that may be quite expensive for the server, and
indeed, this information may be so expensive to provide that SPARQL
planners may not provide it to keep the cost of planning itself
down. In \cite{TODO}, they demonstrated SPARQL evaluation over TPF,
and further improvements have been made in more recent papers.

Olaf Hartig has a large number of papers that are adjacent to the
present work, including his Ph.D. dissertation \cite{TODO} where he
explores querying Linked Data, i.e. query data that is not contained
in an a-priory defined collection of RDF data, but where a resources
are traversed to compute a result. Hartig provides solid formal
foundations for such query processing, as well as computational
feasibility, soundness and completeness. The most relevant paper of
this work is \cite{TODO}.

Acosta et al. \cite{TODO} has ongoing work named ``SHEPHERD'' to
create a SPARQL processor that takes into account observations done by
the SPARQLES survey \cite{TODO}, to decompose a SPARQL query into
subqueries that is indicated to be easier for the server to evaluate,
and so is well aligned with our goal to ease the load of the
server. It is not quite clear from the brief description in 
\cite{TODO} whether the SHEPHERD system can cache results locally.

Montoya et al. \cite{DBLP:journals/corr/MontoyaSMV15} proposes a query
federation system that takes advantage of client-side data
replication. 
