
\section{State of the art}

The following work serve as a starting point for the present work.

\subsection{Conditional selectivity estimation in triple stores}

Selectivity estimation has been a standard technique for optimization
of query responses to databases for a long time. However, conditional
statistics is rarely employed, as most relational systems assume both
\emph{attribute value independence} and \emph{join
  uniformity}\todo{explain what this means}. These are the assumptions
that made the \langcase{} problem so significant in the SUBLIMA
project, as both assumptions were violated by the mere fact that only
languages can be values (i.e. be objects) of the language attribute
(predicate) and the highly skewed value distribution for
languages. Therefore, these assumptions must be relaxed. For RDF
databases the use of detailed statistics to relax these assumptions is
a recent development and the field is very open to further progress.

\cite{Lv:2009:SEC:1685170.1685590} first considered correlated
properties for SPARQL query optimization. However, they only
considered TBox statements, and so, no detailed statistics of ABox
data was compiled. While this does prove valuable in some cases, it
would do nothing to alleviate the \langcase{} problem.

In SPLENDID\cite{splendid} the authors showed how statistics exposed
using the VoID data description vocabulary\cite{void} could be
exploited for query optimization. The authors exploited the fact that
VoID descriptions may expose the number of distinct subjects or
objects for a given predicate, and therefore partially used
conditional statistics. However, they assumed that the distribution of
subjects and objects was uniform.

\subsection{Statistical Relational Learning}

The idea to use Statistical Relational Learning techniques for
selectivity estimation was first proposed in \cite{selectivityPRM}.

The procedure to do this for RDF databases is also quite clear: First,
a network must be learned from the data, then exposed in a SPARQL
endpoint service description so that it could be retrieved by the
\fedeng{}, where the learned network would be looked up to the
estimate the selectivity of a given graph pattern.

Work has been published on different parts of this problem:
\cite{Lin:2011:LRB:2063016.2063042} demonstrated learning relational
bayesian classifiers from RDF data, while [?] \todo{lost reference}
showed how a probabilistic  relational model could be expressed with
OWL. Finally,  \cite{selectivityPRM} could be adapted to RDF according
to the principles in these two papers.