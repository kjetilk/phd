
\section{State of the art}

The following work serve as a starting point for the present work.

\subsection{Conditional selectivity estimation in triple stores}

Selectivity estimation has been a standard technique for optimization
of query responses to databases for a long time. However, conditional
statistics is rarely employed, as most relational systems assume both
\emph{attribute value independence} and \emph{join uniformity}. For
RDF databases the use of detailed statistics to relax these
assumptions is a recent development.

\cite{Lv:2009:SEC:1685170.1685590} first considered correlated
properties for SPARQL query optimization. However, they only
considered TBox statements, and so, no detailed statistics of ABox
data was compiled.

In SPLENDID\cite{splendid} the authors showed how statistics exposed
using the VoID data description vocabulary\cite{void} could be
exploited for query optimization. The authors exploited the fact that
VoID descriptions may expose the number of distinct subjects or
objects for a given predicate, and therefore partially used
conditional statistics. However, they assumed that the distribution of
subjects and objects was uniform.

\subsection{Statistical Relational Learning}

The idea to use Statistical Relational Learning techniques for
selectivity estimation first originated in \cite{selectivityPRM}.