\section{The papers in context}

This section will put the included papers into context. It also enjoys
the benefit of hindsight, and will therefore address some of the
shortcomings that have surfaced since their publication.

\subsection{Read-write hypermedia}\label{sec:conlapis}

This work is a paper entitled ``The necessity of hypermedia RDF and an
approach to achieve it'' that was presented in the First Linked APIs
workshop at the Ninth Extended Semantic Web Conference. I was the sole
author. The reference is \cite{kjernsmo_lapis_2012}.

%TODO: not the Dagstuhl?

This paper is relevant to the topic of the dissertation for two
reasons: First, it shows how hypermedia may be used to drive a
read-write application, not just read-only, which is the focus of most
of the discussion. Second, it provides some of the basis for
\cite{ldf1}, which is important in the query planner.

The main idea in this paper is to look upon RDF as a natural language
sentence with subject, predicate and object, and that sentence should
then be sufficiently human-readable to be straightforward to
understand when a developer does ``View Source''.

Its main weakness is how it deals with authentication: The basic idea
is that when a client visits a resource, it will be told what
write operations it can perform. However, write operations will rarely
be permitted unless the client has authenticated and been authorised
to do so, and so, it assumes that credentials will be supplied with
the first read request. While the HTTP standard permits clients to
supply credentials with such requests, in practice, it rarely if ever
happens, possibly for privacy reasons. Thus, the server will have to
challenge the client first, but this should not happen on a resource
that the client is permitted to read. Thus, the proposed hypermedia
needs a property so that the client understands what to do if it wants
to do write operations.

\subsection{Design of Experiments}\label{sec:condoe}

This work resulted in a paper that was nominated for Best Paper on the
Evaluation track on the International Semantic Web Conference 2013,
entitled ``Introducing Statistical Design of Experiments to SPARQL
Endpoint Evaluation'', which was co-authored with John
S. Tyssedal. The reference is \cite{kjernsmo_doe_intro}.

To understand the context of this study, I noted none of the
benchmarks that was in use for the evaluation of SPARQL engines could
be used for basic statistical hypothesis testing, and clearly,
hypothesis testing must be a fundamental requirement. Then, I noted
substantial criticism forwarded by
\cite{Duan:2011:AOC:1989323.1989340} and \cite{MontoyaVCRA12}.

However, this study merely scratches the surface. To establish a
Design of Experiment-based methodology would be a multi-year project
in its own right, clearly beyond the scope of this
dissertation. Unfortunately, this also implies that I do not have the
rigorous evaluation that I would preferred to have in this
dissertation, it is unattainable.

\subsection{Pushing complexity down the stack}\label{sec:conpush}

This work resulted in a workshop paper entitled ``Pushing complexity
down the stack'', presented in the Developers Workshop at the
International Semantic Web Conference, by the lead author Gregory Todd
Williams, we were the two authors. The reference is
\cite{williamspushing}.

% TODO: detailed contribution?

This work was the outcome of several hackathons that I had organised
in the Perl community. It is not very well known that Gregory Todd
Williams' SPARQL implementation written in Perl is one of the SPARQL
1.1 reference implementations, and is known to be fully compliant with
the specification. Through our work, and the availability of
traits-based programming in Perl, see \cite{traits}, we had extensive
discussions on how to leave optimisations to lower levels, e.g. the
underlying database engine, when they had the potential to provide a
better plan or cost estimate. While it is certainly possible to do
this without this new programming paradigm, traits made the
methodology much more comprehensible, and so helped the subsequent
work substantially.

This work also resulted in the ``Attean''
framework\footnote{\url{https://metacpan.org/release/Attean}}, also
mainly authored by Gregory Todd Williams with some smaller
contributions from me. It is also available in the Linux distribution
Debian (and derivatives, such as Ubuntu) as \texttt{libattean-perl}.

\subsection{Survey of HTTP caching}\label{sec:consanity}

I committed to do the survey after Jürgen Umbrich challenged me to see
if the standards were actually followed. Rather than just survey the
usage for SPARQL Endpoints, I decided to do it for the breadth of the
Semantic Web. This resulted in a research track paper on the Extended
Semantic Web Conference 2015 entitled ``A Survey of HTTP Caching
Implementations on the Open Semantic Web'' with reference
\cite{kjernsmo_survey_2015} and a technical report
\cite{kjernsmo_add_survey_2015} that elaborates on data reduction
methods and statistical methods.

Although the support for caching headers is not yet widespread, it is
sufficient to perform experiments on real-world data.

\subsection{Philosophy of Science}\label{sec:conphil}

Dissatisfaction with the methodology used in the literature,
specifically benchmarking, to test assertions of performance lead to
exploration of statistical methods in referenced in
Section~\ref{sec:condoe}, and to organise the Empirical 2014 Workshop
at the Extended Semantic Web Conference (which I was unfortunately
prevented from attending myself). Still, frustrated by the slow
progress of the last 16 years, and difficulty with comparing methods
to the more mature natural sciences, I still felt the need to express
a philosophical view of science, which I did by submitting an article
to the ``Negative or inconclusive results in Semantic Web'' at the
Extended Semantic Web Conference 2015. The essay, see
\cite{kjernsmo_noise_2015} was tangential to the Call for Papers, but
the chairs decided that Jacco van Ossenbruggen would interview me in
front of the audience about the contents of this essay.

Again, the essay serves somewhat as an excuse for why the evaluations
in this dissertation does not meet the standards that I would
advocate, it would require many years of concerted effort to meet
those standards. The intention is then to promote a discussion of
epistemological questions, and as such, the essay is intentionally
provocative.  In retrospect, the greatest weakness of the essay is
that I failed to recognise the impact of Frank van Harmelen's call for
studies that can establish laws of the Semantic Web information
universe, as discussed in Section~\ref{sec:evalproblems}. To my
defence, I was only made aware of this keynote by the
reviewer. 
