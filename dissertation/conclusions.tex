

\section[Future Work]{Future work; Discussion of Other Problems}

As noted in Section~\ref{sec:motivation}, it is clear that evaluating
arbitrary SPARQL queries are in general unsustainable, and it is
likely that the problems in Section~\ref{sec:perfproblems} are partly
to blame for it. Our work, and most of the related work in
Section~\ref{sec:related} approach this problem from the direction of
moving stress from the server, but it is also clear that another way
to address Problem~\ref{prob:sparqlcomplex} is for the SPARQL engine
to identify queries it would be too heavy to answer already when
parsing or planning, and also to expose that in the service description
so that clients can avoid asking them. The HTTP/2 specification
\cite{rfc7540} has some generic facilities that may be helpful: 

\begin{quote}
\texttt{ENHANCE\_YOUR\_CALM (0xb)}:  The endpoint detected that its peer is
      exhibiting a behavior that might be generating excessive load.
\end{quote}

Problem~\ref{prob:queryeq} is best addressed by creating hashing
algorithms that have a reasonable chance of returning the same digest
if two queries are equivalent. As noted in Section~\ref{sec:relcache},
\cite{kaseicache} can be used for this, and the approach
\cite{papailiou2015graph} took to canonical labelling may be extended
to address this problem

Problems~\ref{prob:unknowndist},~\ref{prob:unknownconddist}~and~\ref{prob:largestats}
has a promising approach in \cite{5767868}.

I have not approached any of the problems of
Section~\ref{sec:semproblems}, mainly because I think it is an open
question which of these approaches will be adopted in practice, and
whatever the solution may be, their designers should make sure they
are orthogonal to the other problems discussed here. Thus, it was
natural to consider them out of scope. 

Clearly, Problem~\ref{prob:lawsfuture} is a very interesting one, with
important implications for evaluation, but also because it shifts the
focus of our attention. In this work, the methodology used to study
Problem~\ref{prob:sanity} could be used to gain such understanding,
but at present, the field is too immature to attempt the formulation
of any laws. As noted in Section~\ref{sec:history}, this study is
partly motivated by my subjective experience of why the Web was so
successful. Clearly, this should be studied more objectively, but I am
not aware of any such study. Such study could also be applied to see
if the experiences from the emergence of the Web can be applied to
accelerating the Semantic Web, like I subjectively propose.

Problem~\ref{prob:tpf} was along with
problems~\ref{prob:sparqlcomplex},~\ref{prob:endpointunpred}~and~\ref{prob:syntacticcache}
part of the motivation of \cite{ldf1} and \cite{verborgh2014querying},
and this builds in part on the insight I contributed when addressing
Problem~\ref{prob:lapis}. Their solution was to define an ontology to
describe an interface that can answer a single triple pattern, but
since it is hypermedia, it can be described in the messages of the
protocol. However, the solution in \cite{verborgh2014querying} does
not take into account Problem~\ref{prob:microcontroller}, so query
planning that shifts the burden to a cost-model driven balance between
server, proxy and client is an interesting direction. In this context,
it should be noted that HTTP/2 \cite{rfc7540} is intended to make
multiple requests less expensive, and it is interesting to study the
practical impact of this as it could impact the cost of making many
HTTP requests substantially.

I believe that the core of Problem~\ref{prob:dynaprog} lies in the
prevalence of statically typed languages. From an engineering
perspective, static typing seems like a good idea, since errors should
surface at compile time. However, it seems like it comes at a high
cost: Programmers seem incapable of thinking beyond the constraints it
sets. Nevertheless, the theoretical framework that seems to solve the
problem was proposed in 1991 in the book ``The Art of the Metaobject
Protocol'' \cite{kiczales1991art}. Perl has a mature implementation of
the metaobject protocol, and the topic has been discussed on several
hackathons, and some code has been written, but the solution seems
elusive.

Finally, the problems~\ref{prob:badphil}~and~\ref{prob:badstats}
should prompt an elaborate approach to understand and apply the
practical implications of contemporary philosophy of
science. \cite{Mayo2005-MAYEAP} proposes a framework that defines what
constitutes a \emph{severe} test, and one possible direction is to
apply this to query endpoint evaluation.

\section{Conclusions}\label{sec:conclusions}

\begin{itemize}
\item Hypermedia is one key to developing applications on the Semantic
  Web.
\item Traits ease the task of developing experimental systems and
  usage of optimisations in underlying systems.
\item While the adoption of caching and conditional requests as defined
  in HTTP is not widespread, the adoption is sufficient to start using
  them in practical systems.
\item Evaluation methodologies suffer from a poor foundation in
  philosophy of science, and research into the epistemology of the
  field is important.
\item The author admits that the study is lacking in terms of
  objectivity, but notes that the problems are of a general nature.
\item Statistical Design of Experiments provides a path to a critical
  practice of evaluations, needed to improve the foundations of the
  field.
\end{itemize}
