\section{Problem description}

The main topic of the present dissertation is SPARQL
Federation\todo{further discuss different interesting directions with
  SPARQL federation}.

In general, heterogenous data often causes optimization
problems. Since heterogenous data is important in Semantic Web
applications, these problems become important to solve in a Semantic
Web context.

One problem can be motivated from previous implementation experience:
In the SUBLIMA project, described in \ref{sublima}, entries were
typically described has having a language, and in one deployment the
vast majority were written in Norwegian, but a few in English, Danish
or Swedish. With the SPARQL engines of the day, this heterogenity
caused significant performance issues: If the language was joined
first, and it was Norwegian, the query would be slow, whereas if it
was joined last, it would be orders of magnitude faster. For English,
Danish or Swedish, the situation was the opposite. The lesson was
learned and fixed on an \textit{ad hoc} basis, the most restrictive
term must be joined first. This issue will be referred to as the
\langcase in this text as we shall return to it.

In modern centralized SPARQL databases, this is less of a problem, but
the problem rearises for distributed databases.
