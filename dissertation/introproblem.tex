\section{Motivation}

The main topic of the present dissertation is SPARQL
Federation\todo{further discuss different interesting directions with
  SPARQL federation}.

In general, heterogenous data often causes optimization
problems. Since heterogenous data is important in Semantic Web
applications, these problems become important to solve in a Semantic
Web context.

One problem can be motivated from previous implementation experience:
In the SUBLIMA project, described in \cite{sublima}, entries were
typically described has having a language, and in one deployment the
vast majority were written in Norwegian, but a few in English, Danish
or Swedish. With the SPARQL engines of the day, this heterogenity
caused significant performance issues: If the language was joined
first, and it was Norwegian, the query would be slow, whereas if it
was joined last, it would be orders of magnitude faster. For English,
Danish or Swedish, the situation was the opposite. The lesson was
learned and fixed on an \textit{ad hoc} basis, the most restrictive
term must be joined first. This issue will be referred to as the
\langcase{} in this text as we shall return to it.

In modern centralized SPARQL databases, this is less of a problem, but
the problem re-emerges for distributed databases. In the case where
the \fedeng{} has no knowledge of the data on the \indendp, it cannot
optimize the join order, as such details are unavailable. For example,
in the \langcase, if the language annotations were on a different
endpoint than other data, the effects would be disasterous, as if the
Norwegian language resources was first retrieved from one endpoint,
the query would retrieve a large result set, and then return the
result set over the network to the \fedeng, with the high
communication costs incurred, and finally the query passed on to the
next federation member with the bound variables would be large, which
would incur a large communication cost as well as the cost of parsing
the query. 

Sparse statistics is not sufficient to solve the \langcase{}
generally. We require knowledge not only of the statistics of the
predicate, but also on the object for a given predicate.

This is the main motivation of the present dissertation. To investigate
possible solutions, we explore different directions in statistics,
ranging from simple extensions to the VoID vocabulary\cite{void} to
advanced \SRL.