\chapter{Introduction}

The World Wide Web, or just the Web for short, is a well-known global
information space invented by Tim Berners-Lee in 1989, further emerged
in the 1990-ties. It is characterised first and foremost by its
universality, anyone can set up a computer, connect it to the Internet and
start serving data or documents from it, and further adapt it to their purpose. 

The Semantic Web is an extension of the Web that has been under
development since 1997\footnote{The first working group draft of the
  RDF specification is dated 1997-08-01, see
  \url{http://www.w3.org/TR/WD-rdf-syntax-971002/}.}, to extend the
Web with languages for expressing information in a machine processable
form\cite{semwebroadmap}.

The Semantic Web is built on a number of specifications that has been
developed under the auspices of the World Wide Web Consortium. The
core technology is known as Resource Description Framework (RDF). 

The present work sits in the confluence of several contemporary
efforts in the Semantic Web community. The focus is on query answering
with the SPARQL query language, with emphasis on exploiting the World
Wide Web, but it touches upon query federation, hypermedia, empiricial
methods for evaluating performance, standards compliance and even
philosophy of science. 

\section{Contributions}

\subsection{Hypermedia}

\begin{enumerate}
\item A thorough discussion of the implications of Hypermedia
  Types\cite{hypermediatypes} for RDF.
\item A sketch of a vocabulary for read-write RDF hypermedia.
\end{enumerate}

\subsection{Design of Experiments}

\begin{enumerate}
\item Introduction of a path to critical practice of evaluations, that makes
  use of contemporary statistical techniques, to establish a practice
  that can be used to refute assertions on performance.
\item A didactical experiment to help researchers understand the statistics.
\item The novel application of a well established method in
  statistics, rarely used in Computer Science, to SPARQL endpoint
  evaluation.
  
\end{enumerate}

\subsection{Development problems}

\begin{enumerate}
\item A framework to enable the use of low-level optimisations in
  databases.
\item Simplification when implementing experimental features in
  SPARQL.
\item Make it possible to compose custom query planners that doesn't
  require inheritance.
\end{enumerate}

\subsection{Survey of HTTP Caching}

\begin{enumerate}
\item An understanding of actual usage of caching headers (metadata)
  on the open Semantic Web.
\end{enumerate}

\subsection{Philosophy of Science}

\begin{enumerate}
\item A provocation to discuss epistemological questions.
\end{enumerate}
