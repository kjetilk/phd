\documentclass[a4paper,english,10pt]{article}
\usepackage{babel}
\usepackage[latin1]{inputenc}
\usepackage[T1]{fontenc}
\usepackage{times}

\title{Problem statement and suggested project outline}
\author{Kjetil Kjernsmo}

\begin{document}

\maketitle


Query federation has been an active field for some time, but has until
the advent of the Semantic Web not been used in a highly diverse set
of endpoints, commonly they have been under the control of a single
institution. RDF has a triple-based data model for the Semantic Web,
and SPARQL is a standardized query language to query such data.

Query federation with SPARQL has attracted much attention
from industry and academia alike, and four implementations of basic
query federation were submitted to the SPARQL 1.1 Working Group as
input for the forthcoming work. The basic query federation feature was
supported by a large number of group members, and the latest working
draft of the proposed standard was published on June 1st 2010 and is
expected to enter a Last Call review period shortly.

There is a very strong industrial use case for SPARQL query
federation, as it is a straightforward way to combine data from the
large number of SPARQL Endpoints. With the emergence of the Linked
Data Web, URIs serves as ``global keys'', and makes data combination
across data sources trivial if a working SPARQL query federation
engine can be used.

While the basic feature set of the proposed standard can enable users
to create federated queries, it is not of great use as it requires
extensive prior knowledge of both the data to be queried and
performance characteristics of the involved query engines. 

I intend to investigate possible remedies to this problem by using
statistical techniques. 

An important technique execution optimization is known as
``selectivity estimation''. This technique seeks to optimize the order
of which parts of the query is executed. In the greater database
literature, this has been widely studied, but the SPARQL spesific
literature leaves many open questions.

An oft-cited paper is \cite{Stocker:2008:SBG:1367497.1367578}. In that
paper, the authors assume that the selectivity of RDF terms are
independent, admit that this assumption is easily violated and
continue to use a conditional estimate for the object node. This is
clearly flawed, and I have already started writing a first paper to
address this. Thus, it is intended that my first paper will address
open issues on selectivity estimation in the single-database
conditional case.  Several directions can be explored, possibly
involving Bayesian Networks, user configurable common SPARQL patterns,
estimation based on known ontologies, etc.




\end{document}
